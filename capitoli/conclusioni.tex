\chapter{Conclusioni e sviluppi futuri}
\label{chap:conclusioni}
In questa tesi è stato proposto un approccio per addestrare modelli SVC con un ristretto numero di vettori di supporto.
Gli esperimenti effettuati su diversi dataset sintetici evidenzia come il metodo sia promettente:
\begin{itemize}
    \item Per alcuni dataset la riduzione in termini di spazio può essere significativa, con una riduzione in termini di accuratezza non significativa.
    \item Per altri dataset la valutazione dell'approccio è più difficoltosa perché si evidenziano casi in cui la riduzione in spazio non intacca l'accuratezza e altri in cui invece la intacca significativamente.
    \item Esprimendo il budget come percentuale della dimensione del dataset, si nota come valori tra il $5\%$ ed il $7\%$ possano essere delle buone scelte per risparmiare spazio senza sacrificare troppo in termini di efficacia.
\end{itemize}

Gli esperimenti effettuati su dataset di terze parti confermano le osservazioni effettuate sui dataset sintetici. ..................


Il confronto con altri metodi evidenzia come \emph{budgeted SVC}..................


Il lavoro iniziato in questa tesi potrebbe sicuramente essere sviluppato ulteriormente, cercando di espandere o migliorare alcuni aspetti, per esempio:
\begin{itemize}
    \item Si potrebbero utilizzare dataset sintetici di dimensione maggiore e/o dataset con un più alto numero di feature. Si potrebbero utilizzare ulteriori dataset di terze parti con caratteristiche diverse rispetto a quelli già considerati.
    \item Si potrebbero proporre ulteriori indicazioni o dei criteri più solidi per come impostare dei valori di budget a priori, per esempio in base alle caratteristiche dei dati disponibili.
    \item Si potrebbero incorporare delle proposte presentate in letteratura, per esempio per limitare l'effetto del rumore o per rendere il problema trattabile con altri metodi risolutivi.
    \item Si potrebbe investigare un eventuale relazione tra \emph{budget} e rumore, per capire se una quantità molto stringente di \emph{budget} corrisponda ad una maggior robustezza rispetto a dati erroneamente etichettati.
    \item ....
\end{itemize}

