\chapter{Conclusioni e sviluppi futuri}
\label{chap:conclusioni}
In questa tesi è stato proposto l'approccio \emph{budgeted SVC} per addestrare classificatori con un ristretto numero di vettori di supporto.
Gli esperimenti effettuati su diversi \emph{dataset} sintetici evidenziano come il metodo sia promettente. 
Le motivazioni sono esposte nell'elenco seguente.
\begin{itemize}
    \item Per alcuni \emph{dataset} facili, la riduzione in termini di spazio può essere significativa e non penalizzante in termini di accuratezza.
    \item Per altri \emph{dataset} difficili, l'accuratezza subisce un peggioramento significativo al diminuire del \emph{budget}, fino al punto in cui per riduzioni importanti l'accuratezza ritorna ad essere buona. Tuttavia, per i modelli peggiori le soluzioni restituite dal risolutore non sono ottime.
    \item Esprimendo il \emph{budget} come percentuale della dimensione del \emph{dataset}, si nota come valori tra il $5\%$ ed il $7\%$ della dimensione del \emph{dataset} possano essere delle buone scelte per risparmiare spazio senza sacrificare troppo in termini di accuratezza.
\end{itemize}

Gli esperimenti effettuati su \emph{dataset} di terze parti sono anche più promettenti dei precedenti risultati:
\begin{itemize}
    \item utilizzando la strategia 1, tutti i modelli con \emph{budget} mostrano almeno l'accuratezza del corrispettivo modello classico;
    \item utilizzando la strategia 2, si evidenzia come la perdita in accuratezza sia proporzionale alla riduzione di \emph{budget}.
\end{itemize}

Il confronto con altri metodi evidenzia come \emph{budgeted SVC} sia in alcuni casi competitivo su \emph{dataset} semplici ma non lo sia invece su \emph{dataset} difficili.
Gli altri due metodi considerati, NSSVM e BSGD SVM,  producono in genere modelli più accurati per valori di \emph{budget} minimi, tra l'$1\%$ e il $5\%$ della dimensione del \emph{dataset} di addestramento.

Per risparmiare ulteriore spazio nella memorizzazione di un modello SVC, anche nella formulazione classica, è stata proposta una procedura di memorizzazione compressa.
Relativamente ai modelli prodotti, si evidenza come questa rappresentazione consenta in parecchi casi una riduzione significativa dello spazio richiesto per memorizzare il modello.

Nonostante i risultati ottenuti siano un buon punto di partenza, la formulazione \emph{budgeted SVC} potrebbe essere studiata ulteriormente, cercando di espandere, migliorare o chiarire alcuni aspetti, elencati di seguito.
\begin{itemize}
    \item Si potrebbe verificare se la variabilità ottenuta su diversi \emph{dataset} generati con gli stessi parametri sia effettivamente dovuta al fatto che il risolutore non fornisca una soluzione ottima. 
    \item Si potrebbero utilizzare \emph{dataset} sintetici di dimensione maggiore e/o \emph{dataset} con un più alto numero di attributi. Si potrebbero utilizzare ulteriori \emph{dataset} di terze parti con caratteristiche diverse rispetto a quelli già considerati.
    \item Si potrebbero incorporare delle proposte presentate in letteratura, per esempio per limitare l'effetto del rumore o per rendere il problema trattabile con altri metodi risolutivi.
    \item Si potrebbe investigare un'eventuale relazione tra \emph{budget} e rumore, per capire se una quantità molto stringente di \emph{budget} corrisponda ad una maggior robustezza rispetto a dati erroneamente etichettati.
\end{itemize}

