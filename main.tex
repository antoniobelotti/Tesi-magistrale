
% Tesi D.S.I. - modello preso da
% Stanford University PhD thesis style -- modifications to the report style

\documentclass[12pt,italian,oneside]{book}
\usepackage{tesi}

% CORSO DI LAUREA:
\def\myCDL{Corso di Laurea magistrale in Informatica}

% TITOLO TESI:
\def\myTitle{Algoritmi di classificazione basati su vettori di supporto per la produzione di modelli succinti}

% AUTORE:
\def\myName{Antonio Belotti}
\def\myMat{Matricola 960822}

% RELATORE E CORRELATORE:
\def\myRefereeA{Prof. Dario Malchiodi}
\def\myRefereeB{Prof. Alberto Ceselli}
\def\myRefereeC{Prof. Michele Barbato}

% ANNO ACCADEMICO
\def\myYY{2022-2023}

% Il seguente comando introduce un elenco delle figure dopo l'indice (facoltativo)
\figurespagetrue

% Il seguente comando introduce un elenco delle tabelle dopo l'indice (facoltativo)
\tablespagetrue

% Package di formato
\usepackage[a4paper]{geometry}		% Formato del foglio
\usepackage[italian]{babel}			% Supporto per l'italiano
\usepackage[utf8]{inputenc}			% Supporto per UTF-8
\usepackage[a-1b]{pdfx}			% File conforme allo standard PDF-A (obbligatorio per la consegna)

% Package per la grafica
\usepackage{graphicx}				% Funzioni avanzate per le immagini
\usepackage{hologo}					% Bibtex logo with \hologo{BibTeX}
%\usepackage{epsfig}				% Permette immagini in EPS
%\usepackage{xcolor}				% Gestione avanzata dei colori


% Package tipografici
\usepackage{amssymb,amsmath,amsthm} % Simboli matematici
%\usepackage{listings}				% Scrittura di codice

% Package ipertesto
\usepackage{url}					% Visualizza e rendere interattii gli URL
\usepackage[pdfa]{hyperref}		    % Rende interattivi i collegamenti interni

\usepackage[italiano,ruled, lined, linesnumbered, longend]{algorithm2e}            % pseudocodice
\DontPrintSemicolon

\usepackage[short]{optidef}         % Problemi di ottimizzazione
\usepackage{etoolbox}
\robustify{\label}

\usepackage[italian]{cleveref}  % aggiunge automaticamente anche "capitolo" o "eq." in base a cosa ci si sta riferendo


% 
% norma di un vettore
\newcommand{\norm}[1]{\|#1\|}

% derivata parziale
\newcommand{\pd}[2]{\frac{\partial{#1}}{\partial{#2}}}

% Vec in grassetto invece che con freccia
\renewcommand\vec{\mathbf}


\makeindex

\begin{document}

\frontespizio
\beforepreface
    % qui ringraziamenti e prefazione volendo
\afterpreface
    % \listoftables
    % \listoffigures
\frontmatter
    \chapter{Introduzione}
\label{chap:introduzione}
L'ottimizzazione delle prestazioni degli algoritmi di apprendimento automatico è una sfida da sempre presente nella ricerca scientifica e nello sviluppo delle tecnologie basate sull'intelligenza artificiale.

Lo studio e l'utilizzo di modelli efficienti è di particolare importanza in tutti quegli scenari caratterizzati da una scarsità di risorse di calcolo o di memorizzazione.
Se è pur vero che le capacità e la disponibilità dell'hardware sono storicamente da sempre in crescita, è vero anche che permangono (e anzi aumentano) scenari in cui la disponibilità di risorse non è scontata, e per cui l'incremento di tali capacità non è fattibile o conveniente.
Si pensi, per esempio, a tutti quegli oggetti di uso quotidiano ormai dotati di computer di bordo, dagli \emph{smartwatch} ai termostati di casa, la gamma di dispositivi per cui si potrebbe sviluppare ed eseguire un algoritmo intelligente è molto vasta e in crescita.

Tra le metodologie di apprendimento automatico più utilizzate, le \emph{support vector machine} (SVM) si sono dimostrate estremamente efficaci nella risoluzione di problemi di classificazione e regressione. 
Tuttavia, per problemi con grandi quantità di dati, i modelli SVM utilizzano un grande numero di \emph{vettori di supporto} (un sottoinsieme di dati di addestramento) per definire una superficie di separazione. 
Questa caratteristica rende i modelli SVM in fase di predizione onerosi in termini di risorse, sia di tempo che di spazio.

Questa tesi si pone l'obiettivo di definire, implementare e valutare sperimentalmente una nuova procedura di addestramento supervisionato per produrre modelli SVM efficienti in grado di utilizzare un numero limitato di vettori di supporto.
In particolare, il lavoro si concentra su problemi di classificazione binaria, introducendo una procedura di addestramento che consente di impostare \emph{a priori} una quantità massima di vettori di supporto.
Nella fase sperimentale si prova a quantificare il guadagno ottenuto in termini di spazio e l'eventuale perdita di accuratezza nelle previsioni rispetto a un modello SVM classico senza nessuna strategia di contenimento del numero di vettori di supporto.

La tesi è organizzata come segue: nel~\Cref{chap:AI_ML} si introduce il campo dell'apprendimento automatico, descrivendone i problemi tipici e alcuni noti approcci per risolverli; nel~\Cref{chap:SVC} si introducono brevemente i \emph{metodi kernel} e si descrivono in dettaglio i modelli \emph{support vector machine}, in particolare per problemi di classificazione; nel~\Cref{chap:sparse_svc} si espone un'analisi della letteratura relativa all'addestramento di modelli SVM parsimoniosi, per poi descrivere la proposta originale di questa tesi, la modellazione \emph{budgeted SVC}; nel~\Cref{chap:esperimenti} si descrivono gli esperimenti effettuati; si conclude infine con una discussione dei risultati ottenuti e dei possibili sviluppi futuri.
\mainmatter
    \chapter{Intelligenza artificiale e apprendimento automatico}
\label{chap:letteratura}


\section{Apprendimento automatico}


\section{Comuni approcci risolutivi}

\subsubsection{K-nearest-neighbour}
\subsubsection{Random forest}
\subsubsection{Logistic regression}
\subsubsection{Clustering}
    \chapter{Support Vector Machine}\label{chap:SVC}
In questo capitolo si descrive genericamente la famiglia dei modelli \emph{support vector machine} (SVM) e in dettaglio la formulazione del modello per risolvere problemi di classificazione (\emph{support vector classifier}, o SVC). 
Nel~\Cref{sec:kernel_methods} si fornisce una breve introduzione ai metodi \emph{kernel} e ai modelli SVC. 
Nel~\Cref{sec:hard_margin_classifier} si descrive la formulazione \emph{hard margin} per problemi di classificazione, che non ammette dati erroneamente etichettati.
Nel~\Cref{sec:soft_margin_classifier} si descrive la formulazione \emph{soft margin} per problemi di classificazione, che ammette la presenza di dati erroneamente etichettati.
Nel~\Cref{sec:kernel_trick} si descrive l'utilizzo del metodo \emph{kernel} per i modelli SVC.
Per concludere, nel~\Cref{sec:svc_limiti} si descrivono in generale le principali limitazioni dei modelli SVM.

\section{Metodi kernel}\label{sec:kernel_methods}
\`E possibile suddividere i modelli di apprendimento automatico in due categorie: modelli lineari e modelli non lineari. 
I modelli lineari vengono utilizzati nei casi in cui si assume che la relazione tra i dati di addestramento $\Vec{x}$ e le etichette $y$ possa essere approssimata in modo accettabile da una funzione lineare $f(\Vec{x}) = \Vec{w}\cdot\Vec{x} + b$, mentre i modelli non lineari sono in grado di approssimare relazioni più complesse.
%I modelli non lineari vengono utilizzati nei casi in cui questa assunzione non si dimostra vera, e per cui serve definire una funzione più complessa per approssimare la relazione.
In genere, produrre un modello lineare è relativamente facile dal punto di vista teorico, ma molti scenari reali esprimono relazioni non lineari. 
Adottare un modello lineare in questi casi porterebbe ad insufficienti capacità di generalizzazione, dovute alla troppa semplicità del modello (\emph{underfitting}). 
\`E quindi necessario sviluppare algoritmi per produrre modelli in grado di esprimere relazioni non lineari, ma utilizzare un modello più complesso potrebbe comunque portare ad insufficienti capacità di generalizzazione. 
Questo problema si potrebbe verificare nel caso in cui il modello si adatti troppo fedelmente ai dati di addestramento (\emph{overfitting}, descritto nel~\Cref{sec:bias_variance_tradeoff}).

Si vorrebbe quindi un modello sufficientemente complesso per approssimare la relazione espressa dai dati, ma allo stesso tempo non troppo complesso da soffrire di \emph{overfitting}; si vorrebbe la semplicità di un modello lineare con le capacità di modellazione di un modello non lineare. 

I \emph{metodi kernel}~\cite{2007_kernel_methods} sono una famiglia di approcci per risolvere vari problemi tipici dell'apprendimento automatico. La particolarità sta nel fatto che non si tratta di metodi creati \emph{ad hoc}, ma di estensioni di algoritmi lineari già noti, potenziati per poter esprimere relazioni più complesse.
L'idea alla base dei metodi \emph{kernel} è di immergere i dati dallo spazio originale $\mathcal{X}$ ad un nuovo spazio $\mathcal{H}$, chiamato spazio delle \emph{feature}, di dimensionalità maggiore rispetto ad $\mathcal{X}$, potenzialmente infinita, utilizzando la trasformazione non lineare $\Phi: \mathcal{X} \rightarrow \mathcal{H}$.
Nello spazio delle \emph{feature} i dati diventano linearmente separabili, ed è quindi possibile utilizzare un modello lineare espresso esclusivamente in termini di prodotto scalare tra elementi in $\mathcal{H}$, ma che esprime una relazione non lineare in $\mathcal{X}$.

Questo procedimento è però costoso dal punto di vista computazionale: va calcolata $\Phi$ per ogni punto e poi va calcolato il prodotto interno tra tutte le coppie di elementi.
I metodi \emph{kernel} evitano di utilizzare esplicitamente la funzione $\Phi$, utilizzando invece una funzione
%Nei metodi \emph{kernel} lo spazio delle \emph{feature} è un \emph{reproducing kernel Hilbert space} (RKHS) \cite{}: questo consente di esprimere il prodotto scalare tra elementi appartenenti ad $\mathcal{H}$ con una funzione espressa in termini di prodotto scalare tra elementi di $\mathcal{X}$. Esiste quindi la funzione 
\begin{equation*}
    K: \mathcal{X} \times \mathcal{X} \rightarrow \mathbb{R} 
\end{equation*}
per cui per ogni $\Vec{x}, \Vec{z} \in \mathcal{X}$ vale
\begin{equation*}
    K(\Vec{x}, \Vec{z}) = \Phi(\Vec{x}) \cdot \Phi(\Vec{z}).
\end{equation*}
Questa funzione $K$ è chiamata \emph{funzione kernel}.
Un metodo \emph{kernel} è un algoritmo di addestramento che riesce a sfruttare una funzione \emph{kernel} nelle fasi di addestramento e predizione, rendendo in questo modo non necessario l'utilizzo di $\Phi$, perché il prodotto scalare delle immagini $\Phi(\Vec{x}) \cdot \Phi(\Vec{z})$ è funzione del prodotto scalare tra $\Vec{x}, \Vec{z}$ nello spazio originale.
Questo calcolo è possibile perché $\mathcal{H}$ è un \emph{reproducing kernel Hilbert space} (RKHS)~\cite{RKHS}.

Con un metodo \emph{kernel} si cerca di sfruttare la semplicità di un modello lineare nello spazio delle \emph{feature} ma in grado di modellare relazioni complesse nello spazio originale.
Il~\Cref{sec:kernel_trick} riporta una spiegazione più approfondita dell'applicazione del metodo \emph{kernel} ai modelli SVC.

Tra i metodi \emph{kernel} più noti, l'approccio forse più importante è quello relativo alla famiglia delle \emph{support vector machine}, che include varianti per risolvere sia problemi di classificazione che di regressione.
Introdotti inizialmente per problemi di classificazione, i modelli SVC estendono il concetto di \emph{maximal margin classifier}. 
L'addestramento di un modello SVC ha come obiettivo quello di trovare un iperpiano che separa i punti appartenenti a due classi diverse con il massimo margine di separazione ottenibile. 
Questo approccio era inizialmente limitato a problemi lineari senza ammissione di dati erroneamente etichettati. 
Nei primi anni novanta, con l'introduzione del metodo \emph{kernel}~\cite{1992_hardmargin_svm} prima e con l'introduzione della formulazione \emph{soft margin}~\cite{1995_svm} poi, i modelli SVC divennero delle buone opzioni per risolvere problemi di classificazione non lineari e con dati erroneamente etichettati, riscuotendo un buon successo e ottenendo in alcuni casi \emph{performance} comparabili e spesso superiori ad altri modelli avanzati. 

Si trovano in letteratura diversi approcci che modificano la formulazione SVM originale, come per esempio \emph{$\nu$-svm}\cite{2000_nu_svm} o  \emph{p-svm}\cite{2001_p_svm}.
In questa tesi si presterà particolare attenzione alla formulazione originale per risolvere problemi di classificazione.

Le formulazioni dei modelli SVC esposte nei prossimi paragrafi sono estratte in gran parte da~\cite{1995_svm,svm_tutorial,elements-of-statistical-learning,RKHS}.

\section{Classificazione \emph{hard margin}}\label{sec:hard_margin_classifier}
Si ipotizza di avere un insieme di dati $\mathcal{X} = \{\Vec{x}_i, i=1,\dots,m\}$ contenente $m$ vettori $d$-dimensionali, $\Vec{x}_i \in \mathbb{R}^d$. 
Ogni $\Vec{x}_i$ ha associata un'etichetta $y_i \in \{-1, +1\}$: si considerano quindi problemi di classificazione binaria.
%
L'insieme $\mathcal{X}$ è linearmente separabile se esistono un vettore $\Vec{w} \in \mathbb{R}^d$ e uno scalare $b \in \mathbb{R}$ (chiamato \emph{bias}) che identificano un iperpiano $\Vec{w}\cdot \Vec{x} +b=0$ in grado di separare perfettamente le due classi di dati.
\begin{figure}
    \centering
    \includegraphics[width=0.5\linewidth]{img/dati_linearmente_separabili.pdf}
    \caption[Esempio problema classificazione binaria linearmente separabile.]{Esempio di dati per un problema di classificazione binaria in cui le classi (identificate dai due colori rosso e blu) sono linearmente separabili. Sono illustrate alcune delle possibili rette in grado di separare correttamente tutti gli esempi della stessa classe. }
    \label{fig:dati_linearmente_separabili}
\end{figure}
Con $\Vec{w}$ e $b$ noti, si definisce il classificatore 
\begin{equation*}
    h(\Vec{x}) = \sign(\Vec{w}\cdot \Vec{x} +b).
\end{equation*} 
Per insiemi di dati linearmente separabili esistono infiniti iperpiani che li separano; l'interesse è di trovare un iperpiano con cui eseguire predizioni accurate su nuovi dati. 
Si riporta in~\Cref{fig:dati_linearmente_separabili} un esempio di insieme di dati con alcuni possibili superfici di separazione. 
Per costruire un buon classificatore, in grado di generalizzare su dati mai visti nella procedura di addestramento, si vorrebbe identificare l'iperpiano che massimizza il margine di separazione tra le due classi, ovvero l'iperpiano che massimizza la distanza tra l'iperpiano stesso e i punti più vicini di ogni classe.
La superficie di separazione deve essere equidistante rispetto agli esempi più vicini di ogni classe.

Per configurare un problema trattabile è necessario quantificare il margine.
Sia $d$ la distanza tra la superficie di separazione e il punto più vicino etichettato come positivo. 
Dato che si vorrebbero separare al meglio le due classi, la superficie di separazione dovrà trovarsi alla stessa distanza anche rispetto al punto più vicino etichettato come negativo.
In~\Cref{fig:optimal_separation_margin} si può vedere un esempio per un insieme di dati in due dimensioni.
Il margine è quindi $M=2d$, ma è possibile trasformarlo in una forma analitica più conveniente per configurare il problema.
Si definiscono a tale scopo i vincoli 
\begin{equation*}\label{eq:svc:hardmargin:margin_1}
    \Vec{w}\Vec{x}_i + b \geq 1 \quad \text{se} \quad y_i = +1 \quad  i=1,\dots,m,
\end{equation*}
\begin{equation*}
    \Vec{w}\Vec{x}_i + b \leq -1 \quad \text{se} \quad y_i = -1 \quad i=1,\dots,m,
\end{equation*}
che forzano la superficie di separazione ad essere equidistante dagli iperpiani su cui giacciono i punti più vicini di ogni classe.
% Dividendo entrambi i lati delle disequazioni per $M$
% \begin{equation*}
% \frac{1}{M}\Vec{w}\Vec{x}_i + \frac{1}{M}b \geq 1,
% \end{equation*}
% \begin{equation*}
% \frac{1}{M}\Vec{w}\Vec{x}_i + \frac{1}{M}b \leq -1,
% \end{equation*}
% otteniamo come iperpiano di classificazione $\frac{1}{M}\Vec{w}\Vec{x}_i + \frac{1}{M}b = 0,$ che è equivalente a $\Vec{w}\Vec{x}_i + b = 0$. Il parametro $M$ è quindi in genere fissato ad $1$.
Dato che $y_i\in\{-1,1\}$, i vincoli precedenti possono essere combinati in
\begin{equation*}
    y_i(\Vec{w}\Vec{x}_i + b) \geq 1 \quad i=1, ..., m.
\end{equation*}
Considerando i punti per cui il vincolo precedente è un'uguaglianza: i punti positivi saranno sull'iperpiano $H1: \vec{w}\cdot \vec{x} + b=+1$, con vettore normale $\vec{w}$ e distanza dall'origine $|1-b|/\norm{\vec{w}}$, mentre i punti negativi saranno sull'iperpiano $H2: \vec{w}\cdot \vec{x} + b=-1$, con vettore normale $\vec{w}$ e distanza dall'origine $|-1-b|/\norm{\vec{w}}$. Si può dimostrare che la distanza $2d$ è equivalente a $2/\norm{\vec{w}}$~\cite{svm_tutorial}.

L'iperpiano ottimo $H0: \Vec{w}^*\Vec{x} + b^* = 0$ è l'iperpiano che separa linearmente $\mathcal{X}$ massimizzando il margine di separazione; il simbolo $*$ indica il valore ottimo di una variabile.

I punti $\Vec{x}_i$ per cui $\Vec{w}^*\Vec{x}_i + b^* = \pm 1$ sono chiamati \emph{vettori di supporto}.
\begin{figure}
    \centering
    \includegraphics[width=0.7\linewidth]{img/margine_separazione.pdf}
    \caption[Esempio \emph{maximal margin classifier}.]{Esempio di superficie di separazione con margine massimale per gli stessi dati utilizzati in~\Cref{fig:dati_linearmente_separabili}. I punti cerchiati sono vettori di supporto. $H0$ è la superficie di separazione, mentre $H1$ e $H2$ sono le superfici su cui giacciono i vettori di supporto. Il margine è $M=2d$.}
    \label{fig:optimal_separation_margin}
\end{figure}
%
Massimizzare il margine $2/\norm{\vec{w}}$ equivale a minimizzare $\norm{\Vec{w}}$. 
Trovare la superficie di separazione con margine massimo equivale quindi a risolvere il problema di ottimizzazione
\begin{equation}
\label{eq:svc:hardmargin:primal}
\begin{aligned}
& \min_{\Vec{w},b} && \norm{\Vec{w}} \\
& \textrm{s.t.} && y_i(\Vec{w}\cdot \Vec{x}_i + b) \geq 1, && i=1,\dots,m. \\
\end{aligned}
\end{equation}
Questo problema viene tradizionalmente risolto considerando una formulazione duale, che consente l'impiego del metodo \emph{kernel} più facilmente rispetto alla formulazione primale.

\subsection{Formulazione duale}\label{subsec:hard_margin_dual}
Si riscrive la funzione obiettivo del problema primale in~\Cref{eq:svc:hardmargin:primal} in una forma più conveniente e si esprimono i vincoli in forma normale, ottenendo il problema
\begin{equation}\label{eq:svc:hardmargin:primal_convenient}
\begin{aligned}
& \min_{\Vec{w},b}    && \frac{1}{2}\norm{\Vec{w}}^2 \\
& \textrm{s.t.} && y_i(\Vec{w}\cdot \Vec{x}_i + b) - 1 \geq 0, && i=1,\dots,m, \\
\end{aligned}
\end{equation}
con funzione obiettivo quadratica e vincoli lineari.
%
Applicando il metodo dei moltiplicatori di Lagrange~\cite{optimization_book} si ottiene la funzione lagrangiana
\begin{equation}
\label{eq:svc:hardmargin:lagrangian}
\begin{split}
L_P(\Vec{w},b, \vec{\alpha})  & = \frac{1}{2}\norm{\Vec{w}}^2 - \sum_{i=1}^{m} \alpha_i (y_i(\Vec{w}\cdot \Vec{x}_i +b) -1) =\\
        & = \frac{1}{2}\Vec{w} \cdot \Vec{w} - \sum_{i=1}^{m} \alpha_i y_i(\Vec{w}\cdot \Vec{x}_i +b) -  \sum_{i=1}^{m} - \alpha_i = \\
        & = \frac{1}{2}\Vec{w}\cdot\Vec{w} - \sum_{i=1}^{m} \alpha_i y_i \Vec{w}\cdot \Vec{x}_i - b \sum_{i=1}^{m} \alpha_i y_i + \sum_{i=1}^{m} \alpha_i,
\end{split}
\end{equation}
da minimizzare rispetto a $\Vec{w},b$ e da massimizzare rispetto ad $\Vec{\alpha}$ 
\begin{equation}
\label{eq:svc:hardmargin:max_min}
\begin{aligned}
& \max_{\vec{\alpha}} \min_{\Vec{w}, b} && L_P(\Vec{w},b, \Vec{\alpha}) \\
& \textrm{s.t.} && \alpha_i \geq 0, && i=1,..., m.\\
\end{aligned}
\end{equation}
%
%
Una soluzione ottima per il problema~\Cref{eq:svc:hardmargin:max_min} deve soddisfare le condizioni di Karush-Kuhn-Tucker (KKT)~\cite{svm_tutorial,optimization_book}:
\begin{align}
    \label{eq:svc:hardmargin:kkt1}
    \pd{L_P(\Vec{w},b,\vec{\alpha})}{\Vec{w}} = \Vec{0}, \\[2mm]
    \label{eq:svc:hardmargin:kkt2}
    \pd{L_P(\Vec{w},b,\vec{\alpha})}{b} = 0, 
\end{align}
\begin{align}
    \label{eq:svc:hardmargin:kkt3}
    y_i(\Vec{x}_i\cdot\Vec{w}+b)-1 \geq 0, && i=1,\dots,m,  \\[2mm]
    \label{eq:svc:hardmargin:kkt4}
    \alpha_i \geq 0, && i=1,\dots,m,  \\[2mm]
    \label{eq:svc:hardmargin:kkt5}
    \alpha_i^*(y_i(\Vec{x}_i\cdot\Vec{w}^*+b^*)-1) = 0,  && i=1,\dots,m.
\end{align}
%Il problema~\ref{eq:svc:hardmargin:max_min} è quadratico convesso **fidatevi**.
%
Le condizioni nelle~\Cref{eq:svc:hardmargin:kkt1,eq:svc:hardmargin:kkt2}, ovvero
\begin{align*}
    \pd{L_P}{\Vec{w}} &= \Vec{w} - \sum_{i=1}^{m}\alpha_iy_i\Vec{x}_i = \Vec{0},\\
    \pd{L_P}{b} &=  \sum_{i=1}^{m}\alpha_iy_i =0,
\end{align*}
si riducono a
\begin{equation}
\label{eq:svc_sub1}
\Vec{w} = \sum_{i=1}^{m}\alpha_iy_i\Vec{x}_i,
\end{equation}
\begin{equation}
\label{eq:svc_sub2}
\sum_{i=1}^{m}\alpha_iy_i = 0.
\end{equation}
% representer theorem [Kimeldorf and Wahba,1971] tells us that the optimal solution w∗ of the primal problem can be represented as a linear combination
Sostituendo le~\Cref{eq:svc_sub1,eq:svc_sub2} nella funzione~(\ref{eq:svc:hardmargin:lagrangian}) si ottiene la funzione obiettivo duale
\begin{equation}
\label{eq:svc:hardmargin:dual_obj_fn}
\begin{split}
L_D(\vec{\alpha})  & = \frac{1}{2}\Vec{w}\cdot\Vec{w} - \sum_{i=1}^{m} \alpha_i y_i \Vec{w}\cdot \Vec{x}_i - b \sum_{i=1}^{m} \alpha_i y_i + \sum_{i=1}^{m} \alpha_i =\\
 &= \frac{1}{2}\sum_{i=1}^{m}\alpha_iy_i\Vec{x}_i \cdot \sum_{i=1}^{m}\alpha_iy_i\Vec{x}_i - \sum_{i=1}^{m} \alpha_i y_i \Vec{x}_i \sum_{j=1}^{m}\alpha_jy_j\Vec{x}_j + \sum_{i=1}^{m} \alpha_i =\\
 &= \frac{1}{2}\sum_{i=1}^{m}\sum_{j=1}^{m}\alpha_i\alpha_jy_iy_j\Vec{x}_i\cdot\Vec{x}_j - 
 \sum_{i=1}^{m}\sum_{j=1}^{m}\alpha_i\alpha_jy_iy_j\Vec{x}_i\cdot\Vec{x}_j + \sum_{i=1}^{m} \alpha_i =\\
 &= -\frac{1}{2}\sum_{i=1}^{m}\sum_{j=1}^{m}\alpha_i\alpha_jy_iy_j\Vec{x}_i\cdot\Vec{x}_j + \sum_{i=1}^{m} \alpha_i,
\end{split}  
\end{equation}
espressa esclusivamente in funzione di $\Vec{\alpha}$.
Aggiungendo i rimanenti vincoli di non negatività di $\vec{\alpha}$ si ottiene il problema duale
\begin{equation}\label{eq:svc:hardmargin:wolfe_dual}
\begin{aligned}
& \max_{\vec\alpha}     && \sum_{i=1}^{m}\alpha_i - \frac{1}{2}\sum_{i=1}^{m}\sum_{j=1}^{m}\alpha_i\alpha_jy_iy_j\Vec{x}_i\cdot\Vec{x}_j\\
& \textrm{s.t.}     && \sum_{i=1}^{m} \alpha_iy_i = 0, \\
&                   && \alpha_i \geq 0 && i=1,\dots,m. \\
\end{aligned}
\end{equation}
%
Una volta risolto il problema duale~\Cref{eq:svc:hardmargin:wolfe_dual} all'ottimo, si otterranno i valori ottimali per i moltiplicatori lagrangiani $\alpha_1^*, ..., \alpha_m^*$.
Ricordando che $\Vec{w} = \sum_{i=1}^{m}\alpha_iy_i\Vec{x}_i$, è possibile calcolare l'ottimo 
\begin{equation}\label{eq:representer_w} %??
\Vec{w}^* = \sum_{i=1}^{m}\alpha_i^*y_i\Vec{x}_i.
\end{equation}
Il vettore $\Vec{w}^*$ è definito dai i dati di addestramento $\Vec{x}_i, y_i$ con un corrispettivo moltiplicatore lagrangiano $\alpha_i > 0$: i vettori di supporto. 
Questi punti definiscono il modello: se si rimuovessero tutti i punti che non sono vettori di supporto e si ripetesse l'addestramento, si otterrebbe lo stesso risultato.
Tutti gli $\Vec{x}_i$ associati ad un moltiplicatore lagrangiano ottimo nullo $\alpha_i=0$ non influiscono sulla superficie di separazione.
Per comodità, si definisce l'insieme
\begin{equation*}
    S = \{i \in \{1,\dots,m\} :\alpha_i >0\},   
\end{equation*}
che contiene tutti gli indici corrispondenti a vettori di supporto.

I modelli SVC si possono inserire nella categoria degli \emph{instance based learner}, dato che eseguono predizioni utilizzando un sottoinsieme dei dati di addestramento.

Per costruire il classificatore $h(\Vec{x}) = \sign(\Vec{w}\cdot \Vec{x} +b)$ rimane ancora da calcolare $b$.  Dalla condizione di KKT in~\Cref{eq:svc:hardmargin:kkt5}, la soluzione ottima soddisferà il vincolo 
\begin{equation*}
\alpha_i^*(y_i(\Vec{w}^*\cdot\Vec{x}_i+b^*)-1)=0.
\end{equation*}
Per ogni $i \in S$, ovvero per ogni vettore di supporto, il vincolo precedente implica  
\begin{equation*}
y_i(\Vec{w}^*\cdot\Vec{x}_i+b^*) -1 = 0,
\end{equation*}
dato che $y_i \in \{-1,1\}$, che equivale a
\begin{equation*}
\Vec{w}^*\cdot\Vec{x}_i+b^*=y_i.
\end{equation*}
Pertanto,
\begin{equation*}
b^*=y_i - \Vec{w}^*\cdot\Vec{x}_i.
\end{equation*} 
Nella pratica, per ottenere un valore numericamente più stabile, si calcola la media dei $b^*$ calcolati su tutti i vettori di supporto.

Ricapitolando, risolvendo il problema duale in~\Cref{eq:svc:hardmargin:wolfe_dual} otteniamo i moltiplicatori lagrangiani ottimi $\alpha_1^*, ..., \alpha_m^*$ che consentono di calcolare $\Vec{w}^*$ come combinazione lineare dei vettori di supporto e in seguito di calcolare $b^*$. 
Tutti gli altri dati di addestramento non influiscono sulla soluzione. 
A questo punto, è costruito il classificatore $h(\Vec{x}) = \sign(\Vec{w}^*\cdot \Vec{x} +b^*)$.
La predizione della classe di un nuovo esempio $\Vec{x}_\text{test}$ sarà ottenuta calcolando 
\begin{align*}
h(\Vec{x}_\text{test})  &= \sign(\Vec{w}^*\cdot\vec{x}_{text} +b^*) =\\
                        &= \sign\left(\sum_{i\in S} \alpha_i^*y_i \vec{x}_i \cdot \vec{x}_{text} + b^*\right)
\end{align*}

I modelli ottenuti risolvendo il problema~\Cref{eq:svc:hardmargin:wolfe_dual} sono chiamati \emph{hard margin} perché non ammettono la presenza di anomalie nei dati: tutti gli esempi della stessa classe devono necessariamente essere nello stesso semispazio identificato dall'iperpiano di separazione ottimo. Il~\Cref{sec:soft_margin_classifier} presenterà la formulazione \emph{soft margin}, che tollera dati anomali (ma può subirne comunque gli effetti negativi). 

\section{Classificazione \emph{soft margin}}\label{sec:soft_margin_classifier}
La formulazione \emph{hard margin} funziona solo su dati linearmente separabili, il che rende il modello troppo rigido per essere applicato su dataset tratti da problemi reali, che spesso esibiscono relazioni più complesse. 
Anche nel caso di dati linearmente separabili un modello \emph{hard margin} potrebbe esibire cattive capacità di generalizzazione a causa di un margine negativamente influenzato da pochi dati erroneamente etichettati, \emph{outlier} o rumore.
Si riportano due esempi di questi scenari in~\Cref{fig:svc:softmargin:casi_che_hardmargin_non_risolve}.

\begin{figure}
    \centering
    \includegraphics[width=\linewidth]{img/casi_dove_hardmargin_va_male_o_non_va.pdf}
    \caption[Esempio problemi non risolvibili con \emph{maximal margin classifier}.]{A sinistra un esempio di dataset non linearmente separabile, non risolvibile con il modello \emph{hard margin}. A destra un caso linearmente separabile la cui soluzione \emph{hard margin} (linea continua) è fortemente influenzata da un singolo \emph{outlier}. La linea tratteggiata indica la superficie di separazione ideale.
    I dati con segno ``X'' sono erroneamente etichettati.}
    \label{fig:svc:softmargin:casi_che_hardmargin_non_risolve}
\end{figure}
Per generalizzare il modello \emph{support vector classifier} e renderlo tollerante ad errori di classificazione, si riformula il problema di ottimizzazione in~\Cref{eq:svc:hardmargin:primal}, introducendo $m$ variabili di \emph{slack}/scarto $\xi_i \geq 0 \quad i=1,...,m$, una per ogni dato di addestramento. 
Ogni valore $\xi_i$ sarà proporzionale alla distanza tra $\Vec{x}_i$ erroneamente classificato e la superficie di separazione. 
La variabile $\xi_i$ sarà nulla per ogni $\Vec{x}_i$ correttamente classificato. Resta da definire come utilizzare le variabili di scarto per penalizzare la scelta di dati erroneamente classificati. 
In generale si definisce il problema primale
\begin{equation}
\begin{aligned}
& \min_{\Vec{w},b}    && \norm{\Vec{w}} + \frac{C}{p}\sum_{i=1}^{m}\xi_i^p\\
& \textrm{s.t.} && y_i(\Vec{w}\cdot\Vec{x}_i + b) \geq 1 - \xi_i, &&  i=1,\dots,m, \\
&               && \xi_i \geq 0,                 &&  i=1,\dots,m.\\
\end{aligned}
\end{equation}
Il parametro $C$, fissato a priori, determina il grado di tolleranza agli \emph{outlier}, bilanciando la funzione obiettivo originale con la penalità introdotta dalle variabili di scarto.
%
Il valore di $p$ è scelto tra $p=1$ (L1-SVC) e $p=2$ (L2-SVC), perché così facendo si ottiene una funzione obiettivo al più quadratica. 
Viene qui considerata la formulazione L1-SVC:
\begin{equation}
\label{eq:svc:softmargin:primal}
\begin{aligned}
& \min_{\Vec{w},b,\Vec{\xi}}    && \norm{\Vec{w}} + C\sum_{i=1}^{m}\xi_i\\
& \textrm{s.t.} && y_i(\Vec{w}\cdot\Vec{x}_i + b) \geq 1 - \xi_i, &&  i=1,\dots,m, \\
&               && \xi_i \geq 0,                 &&  i=1,\dots,m.\\
\end{aligned}
\end{equation}
Il modello così modificato, viene chiamato \emph{soft margin support vector classifier}.
Si riporta in~\Cref{fig:soft_margin} un esempio di modello \emph{soft margin} per un insieme di dati su due dimensioni.

\begin{figure}
    \centering
    \includegraphics[width=.7\linewidth]{img/soft_margin.pdf}
    \caption[Esempio classificazione \emph{soft margin}.]{Con la formulazione \emph{soft margin} si tollerano esempi dalla parte sbagliata della superficie di separazione ($\xi_i>1$) o troppo vicini alla superficie di separazione ($0<\xi_i<1$). I punti in questione sono rappresentati in figura con un segno ``X''.}
    \label{fig:soft_margin}
\end{figure}



\subsection{Formulazione duale}\label{subsec:soft_margin_dual}
Il procedimento per ricavare la formulazione duale è analogo al caso \emph{hard margin}.
Si riscrive la funzione obiettivo del problema~\Cref{eq:svc:softmargin:primal} e si esprimono i vincoli in forma normale, ottenendo il problema
\begin{equation}
\label{eq:svc:softmargin:primal_convenient}
\begin{aligned}
& \min_{\Vec{w},b,\Vec{\xi}}    && \frac{1}{2}\norm{\Vec{w}}^2 + C\sum_{i=1}^{m} \xi_i \\
& \textrm{s.t.} && y_i(\Vec{w}\cdot \Vec{x}_i + b) - 1 + \xi_i \geq 0, && i=1,\dots,m, \\
&               && \xi_i \geq 0,  && i=1,\dots,m.
\end{aligned}
\end{equation}
%
Si applica il metodo dei moltiplicatori di Lagrange, ottenendo la funzione lagrangiana
\begin{equation}
\label{eq:svc:softmargin:lagrange_fn}
\begin{split}
L_P(\Vec{w},b, \Vec{\xi}, \Vec{\alpha}, \Vec{\mu}) = & \frac{1}{2}\norm{\Vec{w}}^2 + C\sum_{i=1}^{m} \xi_i - \sum_{i=1}^{m} \alpha_i (y_i(\Vec{w}\cdot \Vec{x}_i +b) -1 +\xi_i) - \sum_{i=1}^{m}\mu_i\xi_i =\\
        = & \frac{1}{2}\Vec{w}\cdot\Vec{w} + C\sum_{i=1}^{m} \xi_i - \sum_{i=1}^{m} \alpha_i y_i(\Vec{w}\cdot \Vec{x}_i +b) -  \sum_{i=1}^{m} - \alpha_i  \\ & - \sum_{i=1}^{m} \alpha_i\xi_i - \sum_{i=1}^{m}\mu_i\xi_i= \\
        = & \frac{1}{2}\Vec{w}\cdot\Vec{w} + C\sum_{i=1}^{m} \xi_i - \sum_{i=1}^{m} \alpha_i y_i \Vec{w}\cdot \Vec{x}_i - b \sum_{i=1}^{m} \alpha_i y_i \\ & + \sum_{i=1}^{m} \alpha_i - \sum_{i=1}^{m} \alpha_i\xi_i - \sum_{i=1}^{m}\mu_i\xi_i,
\end{split}
\end{equation}
da minimizzare rispetto a $\Vec{w},b,\Vec{\xi}$ e da massimizzare rispetto ad $\Vec{\alpha}, \Vec{\mu}$ 
\begin{equation}
\label{eq:svc:softmargin:max_min}
\begin{aligned}
& \max_{\Vec{\alpha}, \Vec{\mu}} \min_{\Vec{w}, b, \Vec{\xi}} && L_P(\Vec{w},b, \Vec{\xi}, \Vec{\alpha}, \Vec{\mu}) \\
& \textrm{s.t.} && \alpha_i \geq 0,  && i=1,..., m,\\
&               && \mu_i \geq 0,     && i=1,..., m.\\
\end{aligned}
\end{equation}
%
Una soluzione ottima per il problema in~\Cref{eq:svc:softmargin:max_min} deve soddisfare le condizioni di Karush-Kuhn-Tucker:
\begin{align}
    \label{eq:svc:softmargin:kkt1}
    \pd{L_P(\Vec{w},b, \Vec{\xi}, \Vec{\alpha}, \Vec{\mu})}{\Vec{w}} = \Vec{0},  \\[2mm]
    \label{eq:svc:softmargin:kkt2}
    \pd{L_P(\Vec{w},b, \Vec{\xi}, \Vec{\alpha}, \Vec{\mu})}{b} = 0, \\[2mm]
    \label{eq:svc:softmargin:kkt3}
    \pd{L_P(\Vec{w},b, \Vec{\xi}, \Vec{\alpha}, \Vec{\mu})}{\Vec{\xi}} = 0, 
\end{align}
\begin{align}
    \label{eq:svc:softmargin:kkt4}
    \alpha_i^*(y_i(\Vec{x}_i\cdot\Vec{w}^*+b^*)-1+\xi_i^*) = 0,  && i=1,\dots,m, \\[2mm] 
    \label{eq:svc:softmargin:kkt5}
    \mu_i^*\xi_i^* = 0, && i=1,\dots,m, \\[2mm] 
    \label{eq:svc:softmargin:kkt6}
    \alpha_i \geq 0, \mu_i \geq 0, \xi_i \geq 0, && i=1,\dots,m.
\end{align}
Le condizioni nelle~\Cref{eq:svc:softmargin:kkt1,eq:svc:softmargin:kkt2,eq:svc:softmargin:kkt3}, ovvero
\begin{equation*}
    \begin{split}
        \pd{L}{\Vec{w}} & = \Vec{w} - \sum_{i=1}^{m}\alpha_iy_i\Vec{x}_i = \Vec{0},\\
        \pd{L}{b} &=  \sum_{i=1}^{m}\alpha_iy_i = 0,\\
        \pd{L}{\Vec{\xi}} &= C -\sum_{i=1}^{m}\alpha_i - \sum_{i=1}^{m}\mu_i = 0,
    \end{split}
\end{equation*}
si riducono a
\begin{align} 
    \label{eq:svc_hard_sub1}
    \Vec{w} = \sum_{i=1}^{m}\alpha_iy_i\Vec{x}_i,  \\[2mm]
    \label{eq:svc_hard_sub2}
    \sum_{i=1}^{m}\alpha_iy_i = 0, \\[2mm]
    \label{eq:svc_hard_sub3}
    C = \sum_{i=1}^{m}\alpha_i + \sum_{i=1}^{m}\mu_i.
\end{align}
Sostituendo le~\Cref{eq:svc_hard_sub1,eq:svc_hard_sub2,eq:svc_hard_sub3} nella funzione~(\ref{eq:svc:softmargin:lagrange_fn}) si ottiene la funzione obiettivo duale
\begin{equation}
\begin{split}
L_D(\Vec{w},b, \Vec{\xi}, \Vec{\alpha}, \Vec{\mu}) = & \frac{1}{2}\Vec{w}\cdot\Vec{w} + C\sum_{i=1}^{m} \xi_i - \sum_{i=1}^{m} \alpha_i y_i \Vec{w}\cdot \Vec{x}_i - b \sum_{i=1}^{m} \alpha_i y_i \\ & + \sum_{i=1}^{m} \alpha_i - \sum_{i=1}^{m} \alpha_i\xi_i - \sum_{i=1}^{m}\mu_i\xi_i =\\
= & \frac{1}{2}\sum_{i=1}^{m}\alpha_iy_i\Vec{x}_i \cdot \sum_{i=1}^{m}\alpha_iy_i\Vec{x}_i  + \sum_{i=1}^{m}\alpha_i\xi_i + \sum_{i=1}^{m}\mu_i\xi_i \\
& - \sum_{i=1}^{m} \alpha_i y_i  \cdot \Vec{x}_i \sum_{j=1}^{m}\alpha_jy_j\Vec{x}_j + \sum_{i=1}^{m} \alpha_i - \sum_{i=1}^{m} \alpha_i\xi_i - \sum_{i=1}^{m}\mu_i\xi_i =\\
=& \frac{1}{2}\sum_{i=1}^{m}\sum_{j=1}^{m}\alpha_i\alpha_jy_iy_j\Vec{x}_i\cdot\Vec{x}_j + \sum_{i=1}^{m} \alpha_i,
\end{split}  
\end{equation}
espressa esclusivamente in funzione di $\Vec{\alpha}$.
Aggiungendo i rimanenti vincoli si ottiene la formulazione del problema duale
\begin{equation}\label{eq:svc:softmargin:wolfe_dual}
\begin{aligned}
& \max_{\vec{\alpha}}    && \sum_{i=1}^{m}\alpha_i - \frac{1}{2}\sum_{i=1}^{m}\sum_{j=1}^{m}\alpha_i\alpha_jy_iy_j\Vec{x}_i\cdot\Vec{x}_j\\
& \textrm{s.t.} && \sum_{i=1}^{m} \alpha_iy_i = 0, \\
&               && 0 \leq \alpha_i \leq C, && i=1,\dots,m. \\
\end{aligned}
\end{equation}
Come nel caso \emph{soft margin}, $\Vec{w}^*$ e $b^*$ sono ricavati dai vettori di supporto.
Si può notare come la formulazione duale \emph{hard margin} sia sostanzialmente identica alla formulazione duale \emph{soft margin}, con la differenza che in quest'ultima i moltiplicatori lagrangiani hanno un valore limitato al massimo a $C$.

% Dimitris Bertsimas, Jack Dunn, Colin Pawlowski, Ying Daisy Zhuo (2019) Robust Classification. INFORMS Journal on Optimization 1(1):2-34. https://doi.org/10.1287/ijoo.2018.0001 '' Both the primal and dual are convex quadratic optimization problems. Because the dual problem has fewer decision variables, and the majority of these variables tend to be equal to zero or the cost parameter C in the optimal solution, it is typically the problem solved in practice (Friedman et al. 2001). In addition, the dual form is ad vantageous because it allows us to do the kernel trick to learn nonlinear decision rules (Cortes and Vapnik 1995).''



\section{\emph{Kernel trick}}\label{sec:kernel_trick}
I modelli esposti fino ad ora sono dei classificatori lineari e non sono in grado di modellare relazioni più complesse.
Come anticipato nel~\Cref{sec:kernel_methods}, si può rendere il modello SVC non lineare utilizzando una funzione \emph{kernel}.

Si potrebbe pensare di elaborare i dati prima di addestrare un modello, applicando una trasformazione in grado di renderli linearmente separabili in un nuovo spazio.
Considerando per esempio i dati mostrati in~\Cref{fig:kerneltrick:non_lin_sep}, non linearmente separabili nello spazio monodimensionale originale, si potrebbe ipotizzare di applicare una trasformazione utilizzando la funzione $\Phi:\mathbb{R} \rightarrow \mathbb{R}^2,$ con $\Phi(x) = (x, x^2)$, ottenendo un nuovo insieme di dati linearmente separabili nel nuovo spazio, mostrato in~\Cref{fig:kerneltrick:visualized}. 

\begin{figure}
    \begin{subfigure}[t]{.45\textwidth}
        \centering
        \includegraphics[width=\textwidth]{img/non_linearmente_separabili.pdf}
        \caption{Insieme di dati non linearmente separabili nello spazio originale. Il colore identifica la classe di ogni punto.}
        \label{fig:kerneltrick:non_lin_sep}
    \end{subfigure}%
    \hfill
    \begin{subfigure}[t]{.45\textwidth}
        \centering
        \includegraphics[width=\textwidth]{img/kernel_trick_visualized.pdf}
        \caption{I dati della~\Cref{fig:kerneltrick:non_lin_sep}, mappati nel nuovo spazio bidimensionale, sono linearmente separabili (per esempio dalla retta tratteggiata).}
        \label{fig:kerneltrick:visualized}
    \end{subfigure}%
    \caption[Esempio trasformazione di dati in un nuovo spazio.]{Esempio di trasformazione per rendere un insieme di dati linearmente separabili in un nuovo spazio.}
\end{figure}

% \begin{figure}
%     \centering
%     \includegraphics[width=0.5\linewidth]{img/non_linearmente_separabili.pdf}
%     \caption{Esempio in una dimensione di dati non linearmente separabili. Il colore identifica la classe di ogni punto.}
%     \label{fig:kerneltrick:non_lin_sep}
% \end{figure}
% \begin{figure}
%     \centering
%     \includegraphics[width=0.5\linewidth]{img/kernel_trick_visualized.pdf}
%     \caption{I dati della~\Cref{fig:kerneltrick:non_lin_sep}, mappati nel nuovo spazio bidimensionale, sono linearmente separabili (per esempio dalla retta tratteggiata).}
%     \label{fig:kerneltrick:visualized}
% \end{figure}

In generale, si vorrebbero trasformare i dati di addestramento dallo spazio originale $\mathcal{X}$ ad uno spazio delle \emph{feature} $\mathcal{H}$ di dimensioni maggiori, potenzialmente infinite, usando una funzione
\begin{equation}
\label{eq:generic_kernel_mapping}
\Phi(x_i) : \mathcal{X} \rightarrow \mathcal{H}
\end{equation}
in modo da rendere i dati linearmente separabili in $\mathcal{H}$.


Rifacendosi alla formulazione del problema \emph{soft margin} in~\Cref{eq:svc:softmargin:wolfe_dual} è possibile notare come i dati di addestramento compaiano nella funzione obiettivo esclusivamente come prodotto scalare tra di essi. 
Per applicare esplicitamente la trasformazione $\Phi$, si dovrebbe dunque calcolare $\Phi(\Vec{x}_i)\cdot\Phi(\Vec{x}_j)$ per $i=1,\dots,m$, il che renderebbe la procedura costosa dal punto di vista computazionale, se non impossibile ($\mathcal{H}$ di infinite dimensioni). 
Come anticipato nel~\Cref{sec:kernel_methods}, è possibile esprimere il prodotto scalare tra elementi appartenenti ad $\mathcal{H}$ con una funzione espressa in termini di prodotto scalare tra elementi di $\mathcal{X}$. 
Esiste la funzione \emph{kernel}
\begin{equation*}
    K: \mathcal{X} \times \mathcal{X} \rightarrow \mathbb{R} 
\end{equation*}
per cui per ogni $\Vec{x}_i, \Vec{x}_j \in \mathcal{X}$ vale
\begin{equation*}
    K(\Vec{x}_i, \Vec{x}_j) = \Phi(\Vec{x}_i) \cdot \Phi(\Vec{x}_j).
\end{equation*}
Il \emph{kernel trick} consiste nell'utilizzare una funzione \emph{kernel} $K$ in modo che la funzione $\Phi$ non debba essere calcolata esplicitamente. 
La funzione $K$, utilizzando i dati nello spazio originale, calcola un risultato equivalente al prodotto scalare tra i punti trasformati. 
Il valore $K(\Vec{x}_i, \Vec{x}_j)$ può essere interpretato come una misura della ``vicinanza'', o meglio come una una misura di similarità, tra i punti $\Vec{x}_i, \Vec{x}_j$.
Esistono diverse funzioni \emph{kernel}; le più utilizzate sono riportate nell'elenco seguente:
\begin{itemize}
    \item Kernel lineare, calcolato come
    \begin{equation*}
        K(\Vec{x}_1, \Vec{x}_2) = \Vec{x}_1\cdot\Vec{x}_2.
    \end{equation*} 
    Questo \emph{kernel} è in realtà fittizio perché la $\Phi$ corrispondente equivale all'identità $\Phi(\Vec{x}_i)=\Vec{x}_i$. Utilizzare il \emph{kernel} lineare produce un modello lineare nello spazio originale.
    \item Kernel polinomiale, calcolato come
    \begin{equation*}
        K(\Vec{x}_1, \Vec{x}_2) = (\Vec{x}_1\cdot\Vec{x}_2 + 1)^d.
    \end{equation*} 
    Questo \emph{kernel} trasforma i vettori in uno spazio a dimensione finita. L'iperparametro $d$ viene fissato a priori. Utilizzando questo \emph{kernel}, l'iperpiano di separazione nello spazio delle \emph{feature} induce una superficie di separazione nello spazio originale che corrisponde ad una superficie polinomiale di grado al massimo uguale a $d$.
    \item Kernel gaussiano, calcolato come:
    \begin{equation*}
        K(\Vec{x}_1, \Vec{x}_2) = \mathrm{exp}({-\frac{\norm{\Vec{x}_1 - \Vec{x}_2}^2}{2 \sigma^2}}).
    \end{equation*} 
    Questo \emph{kernel} trasforma i vettori in uno spazio di dimensione infinita. L'iperparametro $\sigma$ viene fissato a priori. Utilizzando questo \emph{kernel}, l'iperpiano di separazione nello spazio delle \emph{feature} induce una superficie di separazione nello spazio originale che corrisponde ad una somma di distribuzioni gaussiane multivariate con deviazione standard $\sigma$.
\end{itemize}
%
In generale, le condizioni per far sì che una funzione $K$ sia un \emph{kernel}, derivano da un noto teorema di Mercer\cite{mercer_theorem, RKHS}. Se la funzione \emph{kernel} $K$ è simmetrica, continua e definita semi-positiva, allora esiste una funzione 
\begin{equation}
    \Phi(x) : \mathbb{R}^p \rightarrow \mathcal{H}
\end{equation} 
tale per cui 
\begin{equation*}
    K(\Vec{x}_i, \Vec{x}_j) = \Phi(\Vec{x}_i)\cdot\Phi(\Vec{x}_j).
\end{equation*} 
L'introduzione del \emph{kernel trick} modifica il problema in~\Cref{eq:svc:softmargin:wolfe_dual}, che diventa quindi
\begin{equation}\label{eq:svc:softmargin:wolfe_dual_plus_kernel_trick}
\begin{aligned}
& \max_{\vec{\alpha}}    && \sum_{i=1}^{m}\alpha_i - \frac{1}{2}\sum_{i=1}^{m}\sum_{j=1}^{m}\alpha_i\alpha_jy_iy_jK(\Vec{x}_i, \Vec{x}_j)\\
& \textrm{s.t.} && \sum_{i=1}^{m} \alpha_iy_i = 0, \\
&               && 0 \leq \alpha_i \leq C, && i=1,\dots,m. \\
\end{aligned}
\end{equation}
%
La predizione della classe di un nuovo esempio $\Vec{x}_\text{test}$ sarà ottenuta calcolando 
\begin{align*}
h(\Vec{x}_\text{test})  &= \sign(\Vec{w}^*\cdot\Phi(\vec{x}_{\text{test}}) +b^*) =\\
                        &= \sign\left(\sum_{i\in S} \alpha_i^*y_i \Phi(\vec{x}_i) \cdot \Phi(\vec{x}_{\text{test}}) + b^*\right)=\\
                        &= \sign\left(\sum_{i \in S}\alpha_i^*y_iK(\Vec{x}_i, \Vec{x}_\text{test}) + b^*\right)
\end{align*}

In uno spazio con più dimensioni rispetto all'originale, è sempre possibile trovare un margine di separazione in grado di dividere i dati delle due classi perfettamente. % sempre anche con una sola dimensione in più. e.g. phi([x1]) -> [x1,y1] dove y1 è il label.
Pur essendo un risultato più preciso dal punto di vista del problema di ottimizzazione, potrebbe invece causare \emph{overfitting}. Rimane dunque cruciale la scelta del parametro $C$. 
Per valori di $C$ troppo alti, il modello cercherà di adattarsi troppo fedelmente ai dati di addestramento, creando una superficie di separazione inutilmente complessa nello spazio originale e con pessime capacità di generalizzazione. 
Al contrario, un valore basso di $C$ porterà ad una superficie di separazione più semplice e regolare, bilanciando però con più errori di classificazione. 
Per trovare un valore di $C$ soddisfacente, raggiungendo un buon compromesso tra complessità della superficie di separazione ed errori di classificazione, si utilizzano in genere delle tecniche di \emph{model selection} (viste nel~\Cref{sec:model_selection}).  


\section{Limitazioni}\label{sec:svc_limiti}
I \emph{support vector classifier} presentano alcune limitazioni di cui serve tener conto. 
Sono modelli suscettibili alla presenza di \emph{outlier}, intesi come dati erroneamente classificati o rumore. L'approccio \emph{soft-margin} consente di trattare anche problemi di questo tipo, ma la soluzione trovata potrebbe comunque subire (in misura variabile) l'effetto degli \emph{outlier}, dato che ognuno di questi punti diventerà un vettore di supporto con una relativa variabile di scarto che influirà sul valore della funzione obiettivo. 
Di conseguenza, i modelli SVC addestrati su dataset con rumore hanno in genere performance peggiori rispetto ad altri tipi di modelli pensati specificatamente per dati con molti outlier, per esempio approcci chiamati \emph{robusti} \cite{2019_robust_classification}.
La presenza di rumore nel dataset è comunque un problema comune a tutti i modelli di apprendimento automatico e non riguarda solo i modelli \emph{support vector machine}.

Una seconda limitazione riguarda la scalabilità. La procedura di addestramento considera tutti i dati disponibili e risulta quindi troppo costosa da eseguire su grandi quantità di dati, o su dati con un alto numero di attributi, pur utilizzando un algoritmo di ottimizzazione numerica efficiente, come \emph{sequential minimal optimization} \cite{SMO}.

Una terza limitazione riguarda la selezione dei parametri di addestramento: il costo $C$ e la funzione \emph{kernel}. La scelta ottimale di questi valori richiede in genere molteplici esecuzioni dell'algoritmo di addestramento, moltiplicando i tempi necessari.

Risultano motivati dunque tutti gli approcci che tentano di risolvere queste limitazioni. 


    \chapter{Sparse Support Vector Classifiers}\label{chap:sparse_svc}
Questo capitolo presenta un'analisi dei metodi proposti in letteratura per ottenere modelli SVC con un numero ridotto di vettori di supporto, chiamati \emph{sparse SVC} o in generale \emph{sparse SVM}.
I modelli SVM nella formulazione originale sono già di per sè considerati modelli sparsi, dato che utilizzano una rappresentazione sparsa dei dati di addestramento per eseguire predizioni. Si utilizza comunque il termine \emph{sparse SVM} per indicare approcci che promuovono la sparsità dei vettori di supporto esplicitamente e consentono in alcuni casi di impostare esplicitamente un limite numerico. 
%Gli approcci possibili sono numerosi: 
Nella~\cref{sec:riduzione_post_processing} vedremo alcune proposte per ridurre il numero di vettori di supporto a posteriori. nella ...

Nella~\cref{sec:our_budgeted_svm} si descrive infine la modellazione proposta in questa tesi per produrre \emph{sparse SVC}.

\section{Riduzione post addestramento}\label{sec:riduzione_post_processing}
Tra i primi metodi proposti per ridurre il numero di vettori di supporto, si trovano degli approcci che modificano classificatori già addestrati. Rientra in questa categoria il \emph{reduced set method} ~\cite{reduced_set_method}. Consideriamo un classificatore già addestrato su dati presi dal dominio $\mathcal{L}=\mathcal{R}^k$: la predizione della classe per il punto $\Vec{x}_{new}$ mai visto in fase di addestramento dipende dal risultato di 
\begin{equation}\label{eq:before_reduced_set}
\Vec{w}\cdot\Vec{x}_{new} = \sum_{i=1}^{N_s} \alpha_iy_i\Vec{x}_i \cdot \Vec{x}_{new} = \sum_{i=1}^{N_s} \alpha_iy_iK(\Vec{x}_i, \Vec{x}_{new})
\end{equation}
dove $N_s$ è il numero di vettori di supporto e le $\alpha_i$ sono i moltiplicatori lagrangiani. Il \emph{reduced set method} consiste nell'identificare un insieme di punti $\Vec{z}_a \in \mathcal{L}$ con $a=1,...,N_z$ e dei corrispettivi pesi $\gamma_a \in \mathcal{R}$ che identificano $\Vec{w}^{'} = \sum_{a=1}^{N_z}\gamma_a\cdot\Phi(\Vec{z}_a)$ tale per cui la distanza $p = ||\Vec{w}-\Vec{w}^{'}||$ sia minima. La coppia $\{\gamma_a, \Vec{z}_a\}$ è il \emph{reduced set}, per cui si può rimpiazzare~\cref{eq:before_reduced_set} con
\begin{equation}\label{eq:reduced_set}
\Vec{w}^{'} \cdot \Vec{x}_{new} = \sum_{a=1}^{N_z}\gamma_a \Vec{z}_a\cdot \Vec{x}_{new} = \sum_{a=1}^{N_z}\gamma_a K(\Vec{z}_a, \Vec{x}_{new})
\end{equation}
approssimando il margine di separazione originale.
Idealmente si vorrebbe trovare un \emph{reduced set} di dimensione $N_z << N_s$ per cui le performance degradino di una quantità accettabile, potenzialmente nulla.
Nella prativa, trovare il \emph{reduced set} significa minimizzare la distanza $p$.

Sempre Burges, propone in ~\cite{burges_improving_accuracy} un approccio che combina il reduced set method con un ulteriore \emph{virtual support vector method}. La nuova aggiunta, consiste nell'introdurre delle invarianti note relative al problema trattato, applicando una trasformazione ai vettori di supporto trovati dopo il primo addestramento del modello. Combinando queste due tecniche, gli autori ottengono miglioramento nelle capacità di generalizzazione combinato con un miglioramento delle performance in fase di testing dovuto alla compressione del modello. Questo approccio richiede però una conoscenza specifica relativa al problema trattato che consente di applicare una trasformazione sensata sui vettori di supporto.

\section{Riduzione con addestramento on-line}
I \emph{support vector classifier} possono essere usati anche per addestramento \emph{on-line}, scenario in cui i dati di addestramento sono forniti col passare del tempo. Considerando che la quantità di dati d'addestramento non è nota a priori, risulta fondamentale inserire delle tecniche per contenere il numero di vettori di supporto, che tipicamente cresce al crescere dei dati di addestramento \cite{}. Per questo motivo, esistono numerosi approcci per promuovere la sparsità in contesti \emph{on-line}.

L'algoritmo \emph{perceptron} ~\cite{1958_perceptron} è un algoritmo di addestramento \emph{on-line} originariamente utilizzato per addestrare il modello omonimo (reti neurali con uno solo neurone), ma che può essere usato per addestrare support vector machine. \cite{2003_online_classification_on_a_budget} propone una variante che mantiene un insieme di dimensione variabile dei vettori di supporto. Viene battezzata \emph{variable size cache} perché non viene imposto un limite fisso alla sua dimensione (\emph{variable size} appunto) e gli elementi già presenti possono essere rimossi secondo una strategia adatta, proprio come una \emph{cache} generica. Questi algoritmi di gestione della \emph{cache} verranno riprese in lavori successivi.
Formalmente, al tempo $t$ l'algoritmo riceve un nuovo dato $\Vec{x}_t$ con corrispettivo label $y_t$: si calcola $s_t = \sum_{i=1}^{m} \alpha_iy_iK(\Vec{x}_i, \Vec{x}_t)$ da cui si ricava la predizione $sign(s_t)$. 
Per decidere se $\Vec{x}_t$ debba diventare un nuovo vettore di supporto, si testa la condizione $y_ts_t \leq \beta_t$: in caso positivo si aggiunge $\Vec{x}_t$ come vettore di supporto. La strategia per effettuare questa aggiunta dipende dall'algoritmo specifico. Per prima cosa, serve assegnare un valore ad $\alpha_t$. Si modifica poi il margine di separazione $\Vec{w}_{t+1} = \Vec{w}_t + \alpha_ty_t\Vec{x}_t$. L'ultimo passo opzionale consiste nello scalare la norma del nuovo margine $\Vec{w}_{t+1} \leftarrow c_t\Vec{w}_{t+1}$ per un certo $c_t > 0$. L'algoritmo perceptron nella forma originale usa come parametri $\beta_t=0, \alpha_t=1, c_t=1$. Per limitare il numero di vettori di supporto, una volta aggiunto $\Vec{x}_t$, si rimuovono tutti gli $\Vec{x}_i$ per cui $y_i(\Vec{w} - \alpha_iy_i\Vec{x}_i)\leq \beta$. Questo approccio non impone una dimensione fissa al numero di vettori di supporto e richiede di ricalcolare per ogni vettore di supporto la condizione di mantenimento ogni volta che si aggiunge un nuovo vettore di supporto. 
Sempre in~\cite{2003_online_classification_on_a_budget}, si definisce un \emph{bound} teorico alla dimensione della cache.


Seguendo un approccio analogo, sono state proposte negli anni diverse varianti dell'algoritmo perceptron.
\cite{2005_forgetron}~propone il \emph{forgetron}, una variante che utilizza una strategia di rimozione dei vettori di supporto. Per ogni iterazione, se il modella classifica erroneamente, si aggiornano i vettori di supporto in tre passaggi: si esegue l'algoritmo perceptron classico; si scalano i coefficienti dei vettori di supporto; si rimuove il vettore di supporto con il coefficiente più piccolo.
%
\cite{2007_random_removal}~propone il \emph{random perceptron}, una variante in cui la rimozione di un vettore di supporto viene effettuata casualmente. 
%
\cite{2008_projectron}~propone il \emph{projectron}, una variante che utilizza una strategia di proiezione. 
``The new vector is added to the support set if its
projection onto the linear span of others in the feature space exceeds a predefined threshold,
or otherwise its information is kept through the projection.'' **ok**
%

**merging strategy**

%



\section{Riduzione con addestramento off-line}
Tutti gli approcci visti in precedenza per addestramento on-line possono essere adattati per un approccio off-line, per esempio fornendo artificiosamente il set di addestramento come uno \emph{stream} di dati. Rimane comunque sensato sviluppare approcci dedicati per addestramento off-line, cercando di sfruttare la disponibilità dell'intero dataset come un vantaggio e non come una limitazione.

Molti approcci per la riduzione del numero di vettori di supporto durante l'addestramento \emph{off-line} si basano sull'osservazione che i vettori di supporto erroneamente classificati, ovvero vettori di supporto con le rispettive variabili di \emph{slack} $>0$, influiscono negativamente sul valore della funzione obiettivo, oltre che richiedere spazio per essere memorizzati ed influire sul costo computazione di ogni inferenza. Un approccio per ridurre l'effetto degli \emph{outlier} è la formulazione del problema chiamata L2-svm \cref{eq:l2svc}. La funzione obiettivo è modificata in modo che le variabili di slack non compaiano come norma L1 ma compaiano invece come norma L2.
\begin{equation}
\label{eq:l2svc}
\begin{aligned}
& \min_{w,b}    && \frac{1}{2}\norm{\Vec{w}}^2 + \frac{C}{2}\sum_{i=0}^{m} \xi_i^2 \\
& \textrm{s.t.} && y_i(\Vec{w}\cdot \Vec{x}_i + b) - 1 + \xi_i \geq 0 && i=1,\dots,m \\
&               && \xi_i \geq 0  && i=1,\dots,m 
\end{aligned}
\end{equation}
L'effetto è quello di penalizzare maggiormente la selezione di vettori di supporto con relative variabili di slack di valore $>1$.
Si riportano di seguito alcuni approcci simili, in cui si modifica il termine della funzione obiettivo relativo alle variabili di slack.

~\cite{2005_penalizing_outliers} propone una procedura di addestramento basata sull'osservazione che l'iperpiano di separazione tra le due classi viene reso inutilmente complicato a causa di \emph{outlier}. Per risolvere questi problemi, si riformula la funzione obiettivo del problema primale, introducendo una funzione di penalità non lineare sulle variabili di \emph{slack}, ottenendo il problema~\cref{eq:adaptively_penalizing_outliers}.
\begin{equation}\label{eq:adaptively_penalizing_outliers}
\begin{aligned}
& \min_{\Vec{w}}    && \frac{1}{2}\norm{\Vec{w}}^2 +C \sum_{i=1}^{m}erf(\xi_i, \sigma) \\
& \textrm{s.t.}     && y_i(\Vec{w}\cdot\Phi(\Vec{x}_i) + b) \geq 1 -\xi_i && i=1,\dots,m\\
&                   && \xi_i \geq 0  && i=1,\dots,m 
\end{aligned}
\end{equation}
La funzione di penalità proposta è 
\[
erf(\xi, \sigma) = \frac{2}{\sqrt{\pi}\sigma} \int_{0}^{\xi}e^{\frac{-z^2}{\sigma^2}} dz.
\]
%
Il problema~\cref{eq:adaptively_penalizing_outliers} può comunque essere risolto considerando il duale, rendendo questo approccio relativamente facile da implementare e trattare con tecniche note.

% \cite{2001_ssvm_smooth_svm} propone di rimpiazzare la norma l1 delle variabili di \emph{slack} nella funzione obiettivo del problema primale \emph{soft-margin} con il quadrato della norma l2. Per rendere la funzione obiettivo doppiamente differenziabile (?) utilizza una smooth approximation function (?) così il problema può essere risolto con un fast newton method (?). 

\cite{2006_svm_on_a_budget}



\section{Feature selection durante addestramento}



\section{Budgeted SVC}\label{sec:our_budgeted_svm}
Per limitare il numero di vettori di supporto, si introducono $m$ variabili continue che si comportano però come variabili binarie. Ogni $\Vec{x}_i$ avrà associata una $\gamma_i$ che avrà un valore $\approx1$ se $\Vec{x}_i$ è scelto come vettore di supporto; avrà un valore $\approx0$ in caso contrario.
La prima proposta di riformulazione del problema duale è
\begin{equation}\label{eq:budget_svc:continuous_gamma_formulation}
\begin{aligned}
& \max_{\alpha}    && \sum_{i=1}^{m}\alpha_i - \frac{1}{2}\sum_{i=1}^{m}\sum_{j=1}^{m}\alpha_i\alpha_jy_iy_jK(\Vec{x}_i, \Vec{x}_j) +P(\gamma_1, \dots, \gamma_m)\\
& \textrm{s.t.} && \sum_{i=1}^{m} \alpha_iy_i = 0                   \\
&               && 0 \leq \alpha_i \leq \gamma_iC   && i=1,\dots,m  \\
&               && 0 \leq \gamma_i \leq 1           && i=1, \dots, m\\
&               && \sum_{i=1}^{m} \gamma_i \leq B.                   \\
\end{aligned}
\end{equation}
La funzione $P$ deve essere scelta in modo da forzare le variabili $\gamma_i$ vicino a $0$ o $1$, simulando una variabile binaria.
Una possibile $P$ potrebbe essere 
$$P(\gamma_1, \dots, \gamma_n) = \prod_i -\gamma_i (1 - \gamma_i)$$
oppure 
$$P(\gamma_1, \dots, \gamma_n) = \prod_i \left( \gamma_i^{\gamma_i} (1 - \gamma_i)^{1 - \gamma_i} - 1 \right).$$
%
%
L'utilizzo di variabili $\gamma_i$ continue, consentirebbe l'utilizzo di risolutori basati sul metodo della discesa del gradiente, il che sarebbe vantaggioso a livello pratico sia per le buone performance sia per la facilità di implementazione. (?)
Per semplicità, i primi esperimenti sono eseguiti su una formulazione leggermente diversa che definisce le variabili $\gamma_i$ come variabili booleane
\begin{equation}\label{eq:budget_svc:binary_gamma_formulation}
\begin{aligned}
& \max_{\alpha}    && \sum_{i=1}^{m}\alpha_i - \frac{1}{2}\sum_{i=1}^{m}\sum_{j=1}^{m}\alpha_i\alpha_jy_iy_jK(\Vec{x}_i, \Vec{x}_j)\\
& \textrm{s.t.} && \sum_{i=1}^{m} \alpha_iy_i = 0                   \\
&               && 0 \leq \alpha_i \leq \gamma_iC   && i=1,\dots,m  \\
&               && \sum_{i=1}^{m} \gamma_i \leq B                   \\
&               && \gamma_i \in \{0,1\}.                            \\
\end{aligned}
\end{equation}
Questa scelta forza nella pratica ad utilizzare un risolutore in grado di trattare problemi con variabili intere.


    \chapter{Esperimenti e risultati}
\label{chap:esperimenti}
In questo capitolo si descrivono gli esperimenti effettuati.
Nel~\Cref{sec:exp:dataset} si descrivono gli algoritmi utilizzati per generare i \emph{dataset} sintetici, le caratteristiche dei \emph{dataset} di terze parti e le metriche utilizzate per valutare la ``difficoltà'' di classificazione di un dataset;
nel~\Cref{sec:exp:synth_2d} si descrivono gli esperimenti effettuati su \emph{dataset} sintetici a 2 dimensioni, con e senza rumore;
nel~\Cref{sec:exp:synth_3d} si descrivono gli esperimenti effettuati su \emph{dataset} sintetici a 3 dimensioni senza rumore;
nel~\Cref{sec:exp:real_ds} si descrivono gli esperimenti effettuati su alcuni \emph{dataset} di terze parti utilizzati in letteratura per problemi di classificazione;
nel~\Cref{sec:comparazione_metodi} si confrontano i risultati ottenuti dall'algoritmo proposto in~\Cref{sec:our_budgeted_svm} con alcune implementazioni di algoritmi presenti in letteratura e citati nel~\Cref{chap:sparse_svc}.

\section{Dataset}\label{sec:exp:dataset}
Per gli esperimenti descritti in questo capitolo sono state usate due tipologie di dataset. 
La prima categoria è composta da \emph{dataset} sintetici, generati specificatamente per questo lavoro. 
La seconda categoria è composta da alcuni \emph{dataset} di terze parti utilizzati in letteratura. 
Le due tipologie presentano vantaggi e svantaggi. Nel primo caso risulta molto comodo poter modificare a piacimento le caratteristiche dei dataset, per esempio per valutare algoritmi su \emph{dataset} di difficoltà crescente, oppure con quantità di rumore crescente.
Lo svantaggio è che si potrebbero ottenere dei risultati poco significativi, perché ottenuti su dati generati \emph{ad hoc} e poco confrontabili con altri modelli. 
Per questo motivo ha senso utilizzare anche dei \emph{dataset} di terze parti, spesso utilizzati anche da altri lavori presenti in letteratura.

\subsubsection{Metriche delle difficoltà di un dataset}\label{sec:metriche_dataset}
Per avere a priori un'indicazione della difficoltà dei \emph{dataset} sintetici, intesa come difficoltà nel trovare una superficie di separazione soddisfacente, sono state considerate alcune metriche, in particolare F1 e F1v~\cite{ds_complexity}, descritte nell'elenco seguente.
\begin{itemize}
    \item \textbf{F1} è la sigla per la metrica \emph{Maximum Fisher’s discriminant ratio}. Questa metrica misura la sovrapposizione tra le varie feature per ogni classe.
    La metrica è calcolata come
    \begin{equation*}
        F1=\frac{1}{1+\max_{i=1}^{m}r_{f_{i}}},
    \end{equation*}
    dove $r_{f_{i}}$ è il \emph{discriminant ratio} per la feature $i$.
    Il valore di F1 è compreso nell'intervallo $(0,1]$. Più il valore è vicino ad 1, più il \emph{dataset} è difficilmente classificabile, perché non esistono feature in grado di discriminare le due classi. Al contrario, un valore basso indica la presenza di una \emph{feature} $f$ per cui esiste una superficie di separazione perpendicolare all'asse di $f$ in grado di separare equamente le classi delle etichette.
    \item \textbf{F1v} è la sigla per la metrica \emph{Directional-vector Maximum Fisher’s Discriminant Ratio}. Quantifica quanto separabili siano due classi una volta proiettate su un vettore scelto per rendere i dati separabili.
    Anche il valore di F1v è compreso nell'intervallo $(0,1]$ e valori bassi indicano dati facilmente separabili.    
    Si rimanda a~\cite{ds_complexity} per la definizione completa di F1v.
\end{itemize}

Dal punto di vista pratico, queste metriche sono state calcolate per ogni \emph{dataset} utilizzando la libreria python problexity~\cite{problexity} in versione 0.5.6.


\subsection{Dataset sintetici}
I \emph{dataset} sintetici sono costruiti selezionando dei punti generati casualmente e in seguito etichettati da una funzione.
Per decidere l'etichetta di un punto sono state utilizzate due tipologie di funzioni: una funzione sinusoidale ed una funzione paraboloide.
\begin{itemize}
    \item \textbf{Funzione sinusoidale} Fissando i parametri $\beta,\rho,\theta$, per un vettore $\Vec{x}=[x^{(1)},x^{(2)}]$, $\Vec{x} \in R^2$, l'etichetta $y$ è calcolata con la funzione
    \begin{equation}\label{eq:sinusoid_dataset_lf}
    \textrm{lf}(\Vec{x}) = \sign\left(\frac{1}{(1 + \exp(-\beta(x^{(1)} - 0.5)) + \rho \sin(2\pi\theta x^{(1)})} - x^{(2)}\right).
    \end{equation}
    Questa funzione di etichettatura è utilizzata solo per dati in spazi con due dimensioni. La~\Cref{fig:sinusoid_dataset} mostra alcuni esempi di \emph{dataset} e funzioni di etichettatura al variare dei parametri.

    \item \textbf{Funzione paraboloide} Fissati i parametri $\alpha, x_\text{shift}, y_\text{shift}$, per un vettore $\Vec{x}=[x^{(1)}\dots, x^{(d)}]$, $\Vec{x} \in \mathrm{R}^d$, l'etichetta $y$ è calcolata con la funzione
    \begin{equation}\label{eq:pacman_dataset_lf}
    \textrm{lf}(\Vec{x})= x^{(d)} - \sum_{j=1}^{d-1}\alpha(x^{(j)} - x_\text{shift})^2 - y_\text{shift}.
    \end{equation}
    Il parametro $\alpha$ controlla l'ampiezza del paraboloide, mentre $x_\text{shift}$ e $y_\text{shift}$ traslano il vertice del paraboloide.
    La~\Cref{fig:pacman_dataset} mostra alcuni esempi di \emph{dataset} e funzioni di etichettatura al variare dei parametri.
\end{itemize}
\begin{figure}
    \centering
    \includegraphics[width=.7\linewidth]{img/sinusoid_dataset_param_influence.pdf}
    \caption[Esempio di \emph{dataset} sintetici generati con funzione sinusoidale.]{Esempio di \emph{dataset} sintetici generati con funzione sinusoidale. 
    $\beta$ controlla la pendenza, $\rho$ controlla l'ampiezza, $\theta$ la frequenza.}
    \label{fig:sinusoid_dataset}
\end{figure}
\begin{figure}
    \centering
    \includegraphics[width=.5\linewidth]{img/pacman_dataset_param_influence.pdf}
    \caption[Esempio di \emph{dataset} sintetici generati con funzione paraboloide.]{Esempio di \emph{dataset} sintetici generati con funzione paraboloide. Il parametro $\alpha$ controlla l'ampiezza.}
    \label{fig:pacman_dataset}
\end{figure}
A prescindere dalla funzione di etichettatura utilizzata, la procedura di generazione del \emph{dataset} è descritta nell'\Cref{alg:generazione_dataset_sintetici}.
Questa procedura richiede diversi parametri, descritti nell'elenco seguente.
\begin{itemize}
    \item Il parametro $n$ che identifica la dimensione totale del dataset.
    \item Il parametro \emph{seed} o seme, utilizzato per il generatore di numeri casuali.
    \item Il parametro \emph{test\_size} è la percentuale di $n$ da riservare come insieme di \emph{test}.
    \item Il parametro 
    \begin{equation*}
        p = \frac{\text{numero di elementi con etichetta positiva}}{\text{numero di elementi con etichetta negativa}}
    \end{equation*} 
    regola la proporzione tra esempi negativi ed esempi positivi.
    \item Il parametro $r$ indica la percentuale di elementi della classe positiva e altrettanti esempi della classe negativa da selezionare a caso per poi invertirne l'etichetta.
\end{itemize}
\begin{algorithm}
    \SetAlgoLined
    \KwData{
        $n>0 \in \mathrm{N}$ dimensione del \emph{dataset} desiderata\\ 
        $p \in [0,1]$ per regolare il bilanciamento tra classi\\
        $r \in [0,1]$ per regolare la quantità di rumore\\
        \textit{test\_size} percentuale di dati da ritornare come \emph{test set}\\
        $s$ seme per le operazioni casuali\\
    }
    \KwResult{Dati $X$ ed etichette $y$ suddivisi in addestramento e test}
    Inizializza generatore casuale utilizzando il seme $s$\;
    $X_{pop} \gets$ seleziona a caso $10*n$ sample\;
    $y_{pop} \gets$ etichetta $X_{pop}$ con la funzione di etichettatura\;
    %
    $N_p \gets \lfloor\frac{n}{p + 1}\rfloor$\;
    $N_n \gets n - N_p$\;
    $X, y \gets$ seleziona a caso $N_p$ sample con etichetta positiva e $N_n$ sample con etichetta negativa da $X_{pop},y_{pop}$\;
    seleziona a caso $\lfloor r * N_p \rfloor$\ esempi della calsse positiva ed altrettanti elementi della classe negativa per cui invertire l'etichetta\;
    $X_{test}, y_{test} \gets$ seleziona \textit{test\_size} elementi (con rispettive etichette) a caso come test set\;
    $X_{train} \gets X \setminus X_{test}$\;
    $y_{train} \gets y \setminus y_{test}$\;
\caption{Procedura generica per la generazione di \emph{dataset} sintetico.}
\label{alg:generazione_dataset_sintetici}
\end{algorithm}

% \subsubsection{Funzione sinusoidale}
% Fissando i parametri $\beta,\rho,\theta$, per un vettore $\Vec{x}=\{x_1,x_2\}$, l'etichetta $y$ è calcolata con la funzione
% \begin{equation}\label{eq:sinusoid_dataset_lf}
% lf(\Vec{x}) = \sign\left(\frac{1}{(1 + \exp(-\beta(x_1 - 0.5)) + \rho \sin(2\pi\theta x_1)} - x2\right).
% \end{equation}
% Questa funzione di etichettatura è utilizzata solo per dati in spazi con due dimensioni. La~\cref{fig:sinusoid_dataset} mostra alcuni esempi di \emph{dataset} e funzioni di etichettatura al variare dei parametri.
% \begin{figure}
%     \centering
%     \includegraphics[width=1\linewidth]{img/sinusoid_dataset_param_influence.pdf}
%     \caption{Esempio di \emph{dataset} generati con funzione sinusoidale. 
%     $\beta$ controlla la pendenza, $\rho$ controlla l'ampiezza, $\theta$ la frequenza.}
%     \label{fig:sinusoid_dataset}
% \end{figure}
% \subsubsection{Funzione paraboloide}
% Fissati i parametri $\alpha, x_\text{shift}, y_\text{shift}$, per un vettore $\Vec{x}=\{x_1, \dots, x_n\} \in \mathrm{R}^n$, l'etichetta $y$ è calcolata con la funzione
% \begin{equation}\label{eq:pacman_dataset_lf}
% lf(\Vec{x})= x_n - \sum_{i=1}^{n-1}\alpha(x_i - x_\text{shift})^2 - y_\text{shift}.
% \end{equation}
% Il parametro $\alpha$ controlla l'ampiezza del paraboloide, mentre $x_\text{shift}$ e $y_\text{shift}$ traslano il vertice del paraboloide.
% La~\Cref{fig:pacman_dataset} mostra alcuni esempi di \emph{dataset} e funzioni di etichettatura al variare dei parametri.
% \begin{figure}
%     \centering
%     \includegraphics[width=1\linewidth]{img/pacman_dataset_param_influence.pdf}
%     \caption{Esempio di \emph{dataset} etichettati con paraboloide. Il parametro $\alpha$ controlla l'ampiezza.}
%     \label{fig:pacman_dataset}
% \end{figure}
I parametri per le funzioni di etichettatura utilizzati per la generazione dei \emph{dataset} sono scelti in modo da avere vari livelli di ``difficoltà'' di classificazione, misurata con le metriche esposte in~\Cref{sec:metriche_dataset}.
La difficoltà può essere gradualmente aumentata per esempio stringendo progressivamente il paraboloide o aumentando la frequenza o l'ampiezza della funzione sinusoidale.

La procedura nell'~\Cref{alg:generazione_dataset_sintetici} utilizza delle estrazioni casuali in diversi punti:
\begin{itemize}
    \item per selezionare la popolazione iniziale;
    \item per estrarre i dati di test;
    \item eventualmente nella procedura di introduzione del rumore, per selezionare degli esempi a cui invertire l'etichetta.
\end{itemize}

\subsection{Dataset di terze parti}
I \emph{dataset} di terze parti utilizzati per alcuni esperimenti provengono dalla pagina web della libreria LibSVM\footnote{\url{https://www.csie.ntu.edu.tw/~cjlin/libsvmtools/datasets/binary.html}}.
Sono \emph{dataset} utilizzati in letteratura ma pre-elaborati (scalando e normalizzando gli attributi) e resi disponibili in formato LibSVM~\cite{libsvm}.
Si riportano in~\Cref{tab:uci_datasets} le caratteristiche dei \emph{dataset} utilizzati.
\begin{table}
    \centering
    \begin{tabular}{cccc}
        \toprule
        Nome & Num. dati addestramento & Num. dati \emph{test} & Num. attributi\\
        \midrule
        svmguide1 &  3,089 & 4,000 & 4 \\
        a1a & 1,605	& 30,956 & 123\\
        gisette & 6000 & 1000 & 5000 \\
        \bottomrule
    \end{tabular}
    \caption{Caratteristiche \emph{dataset} di terze parti.}
    \label{tab:uci_datasets}
\end{table}


\section{Impostazione esperimenti}
A prescindere dai \emph{dataset} utilizzati, sono state utilizzate due strategie leggermente diverse per eseguire gli esperimenti; si descrivono nei prossimi paragrafi.

\subsection{Strategia 1}
Il procedimento è descritto di seguito. Per prima cosa si addestra un modello SVM tradizionale senza nessun vincolo sul numero di vettori di supporto. Chiamiamo questo modello $FM$ e chiamiamo $FB$ il numero di vettori di supporto di $FM$.
In seguito, si addestrano una serie di modelli con un \emph{budget} pari ad una frazione di $FB$.
Per ognuno di questi modelli il \emph{budget} è espresso come $B=p*FM$, dove
\begin{equation*}
    p\in\{0.3, 0.4, 0.5, 0.6, 0.7, 0.8, 0.9\}.
\end{equation*}
Nell'~\Cref{alg:esperimenti_1} è descritto lo pseudocodice per la strategia.
\begin{algorithm}
    \SetAlgoLined
    \KwData{\emph{dataset} da analizzare}
    %\KwResult{}
    $FM \gets$ seleziona il miglior modello senza imporre nessun vincolo di budget\;
    $FB \gets$ numero di vettori di supporto del modello $FM$\;
    salva parametri e dettagli di $FM$\;
    \For{$p\gets0.3$ \KwTo $0.9$ \KwBy $0.1$}{
        $B\gets p*FB$\;
        $M \gets$ seleziona il miglior modello con $\text{budget}=B$\;
        salva parametri e dettagli di $M$\;
    }
\caption{Pseudocodice strategia 1.}
\label{alg:esperimenti_1}
\end{algorithm}

Così facendo, è possibile misurare l'accuratezza sui dati di \emph{test} al diminuire del \emph{budget} e paragonarla all'accuratezza del corrispettivo modello classico addestrato sullo stesso dataset.
Per ogni modello, si definisce \emph{score ratio} la quantità
\begin{equation*}
    \text{score ratio} = \frac{\text{accuratezza su dati di test del modello con vincolo sul budget}}{\text{accuratezza su dati di test del corrispettivo modello senza budget}}.
\end{equation*}

Con questi esperimenti è possibile ottenere delle indicazioni su quanto sia possibile ridurre il \emph{budget} senza penalizzare troppo l'accuratezza, limitatamente ai \emph{dataset} considerati e in rapporto alla controparte senza vincolo sul budget.

Per mitigare un eventuale \emph{bias} dovuto alla casualità nella generazione del dataset, si è deciso in alcuni casi di generare tre \emph{dataset} per ogni gruppo di parametri ma utilizzando però un seme per ognuno diverso.
La procedura completa in questo caso è descritta nell'~\Cref{alg:esperimenti_2}.
\begin{algorithm}
    \SetAlgoLined
    \KwData{parametri \emph{dataset} $\gets$ le varie combinazioni di parametri per generare dataset.}    
    \For{$s \gets \text{estrai nuovo seme}$}{
        \For{params \KwIn \text{parametri dataset}}{
            $ds\gets$ genera \emph{dataset} con l'~\Cref{alg:generazione_dataset_sintetici} usando $params$ e $s$\;
            esegui l'~\Cref{alg:esperimenti_1} su $ds$\;           
        }
    }
\caption{Pseudocodice esperimenti con strategia 1 e con ripetizione della generazione dei dataset.}
\label{alg:esperimenti_2}
\end{algorithm}

\subsection{Strategia 2}
Un secondo approccio per eseguire gli esperimenti su un certo \emph{dataset} consiste nell'esprimere il \emph{budget} in funzione alla dimensione del \emph{dataset} invece che in funzione al numero di vettori di supporto di un modello classico.
Questa strategia è descritta di seguito, e come pseudocodice nell'~\Cref{alg:esperimenti_3}.
\begin{algorithm}
    \SetAlgoLined
    \KwData{$ds$ dataset}
    %\KwResult{}
    $m \gets |ds|$\;
    \For{$p$ \KwIn $\{0.01, 0.025, 0.05, 0.075, 0.1, 0.125, 0.15, 0.175, 0.2\}$}{
        $B\gets pm$\;
        $M \gets$ seleziona il miglior modello con $\text{budget}=B$\;
        salva parametri e dettagli di $M$\;
    }
\caption{Pseudocodice strategia 2.}
\label{alg:esperimenti_3}
\end{algorithm}

Per un certo \emph{dataset}, si esprime il \emph{budget} come $B=pm$, dove $m$ è il numero di elementi nel \emph{dataset} e dove
\begin{equation*}
    p\in\{0.01, 0.025, 0.05, 0.075, 0.1, 0.125, 0.15, 0.175, 0.2\}.
\end{equation*}
Così facendo possiamo verificare la bontà dei modelli con un \emph{budget} molto stringente, fino all'$1\%$ dei dati di addestramento, senza essere vincolati al numero di vettori di supporto di un modello classico.

Per gli esperimenti effettuati con questa strategia, i \emph{dataset} sintetici sono stati generati una sola volta con un solo seme per ogni configurazione di parametri.

\subsection{Selezione e valutazione modelli}
Ogni modello è valutato sullo stesso insieme di dati di \emph{test} per ogni dataset.
Nel caso dei dati di terze parti, l'insieme di \emph{test} è già fornito, mentre nel caso dei \emph{dataset} sintetici l'insieme di \emph{test} viene selezionato dopo la procedura di generazione e non viene più modificato.
L'insieme di \emph{test} è scelto casualmente ma mantiene il bilanciamento delle classi dei dati di addestramento.

I migliori iperparametri sono selezionati sui dati di addestramento utilizzando \emph{5-fold cross validation grid search}, con una griglia di parametri descritta in~\Cref{tab:gridsearch_2d}.
\begin{table}
    \centering
    \begin{tabular}{cccc}
        \toprule
        $C$ & \emph{Kernel} & $\gamma$ & d \\
        \midrule
        \multirow{3}{*}{[0.01, 0.1, 1, 10]} & Gaussiano   & [0.0001, 0.001, 0.01, 0.1, 1, 10]   & /\\
                                            \cline{2-4}
                                            & Polinomiale   & / & [2, 5, 10] \\
                                            \cline{2-4}
                                            & Lineare       & / & / \\
        \bottomrule
    \end{tabular}
    \caption{Griglia di iperparametri per selezionare i modelli \emph{budgeted SVC}.}
    \label{tab:gridsearch_2d}
\end{table}
Questi parametri sono riferiti al metodo \emph{budgeted SVC}; nel caso in cui un esperimento sia stato eseguito con una griglia differente, sarà specificato nel paragrafo che lo descrive.

In questo capitolo, in generale, quando si parla di miglior modello si intende un modello selezionato utilizzando \emph{cross-validation} per identificare la miglior combinazione di valori degli iperparametri.

\subsection{Ambiente di esecuzione}
Tutti gli esperimenti sono stati eseguiti su un server con CPU Intel(R) Xeon(R) W-1250P CPU @ 4.10GHz e con 31GB di ram. Il linguaggio di programmazione utilizzato è Python 3.10.10 ed è stato utilizzato il risolutore Gurobi 10.0.1~\cite{gurobi}. 

Per cercare di ridurre il carico computazionale, gli esperimenti effettuati utilizzando la formulazione \emph{budgeted SVC} utilizzano una matrice \emph{kernel} pre-calcolata; le implementazioni di altri metodi presenti in letteratura utilizzati come confronto, invece, calcolano i valori di \emph{kernel} durante l'addestramento.

\section{Esperimenti su dataset sintetici 2D}\label{sec:exp:synth_2d}
Si riportano in questo paragrafo gli esperimenti effettuati sui \emph{dataset} sintetici generati con le configurazioni di parametri riportate nelle~\Cref{tab:parametri_ds_sin,tab:parametri_ds_pacman}.
\begin{table}
    \centering
    \begin{tabular}{ccccc}
        \toprule
         $n$ & \emph{test\_size} & $\beta$ & $\rho$ & $\theta$ \\
        \midrule
        \multirow{10}{*}{1000} & \multirow{10}{*}{0.3} &\multirow{5}{*}{0}  & 0.01  & \multirow{5}{*}{20} \\        
                            &&& 0.025 &     \\        
                            &&& 0.05  &     \\        
                            &&& 0.075 &     \\        
                            &&& 0.1   &     \\
        \cline{3-5}
                &&  95      & 0.2   & 10    \\   
        \cline{3-5}
        &&  \multirow{4}{*}{0}  & \multirow{4}{*}{0.1}  & 1     \\    
                            &&&                       & 2     \\    
                            &&&                       & 5     \\    
                            &&&                       & 1     \\    
        \bottomrule
    \end{tabular}
    \caption{Configurazioni di parametri utilizzati per generare \emph{dataset} bidimensionali con la funzione sinusoidale in~\Cref{eq:sinusoid_dataset_lf}.}
    \label{tab:parametri_ds_sin}
\end{table}
\begin{table}
    \centering
    \begin{tabular}{ccccc}
        \toprule
         $n$ & \emph{test\_size} &$\alpha$ & $x_\text{shift}$ & $y_\text{shift}$ \\
        \midrule
        \multirow{5}{*}{1000} & \multirow{5}{*}{0.3} &1   & \multirow{5}{*}{0.5} & \multirow{5}{*}{0.5} \\
        && 2.5 &\\
        && 5   &\\
        && 7.5 &\\
        && 10  &\\
        \bottomrule
    \end{tabular}
    \caption{Configurazioni di parametri utilizzati per generare \emph{dataset} bidimensionali con la funzione paraboloide in~\Cref{eq:pacman_dataset_lf}.}
    \label{tab:parametri_ds_pacman}
\end{table}

\subsubsection{Esperimenti utilizzando la strategia 1}
In~\Cref{fig:risultati_2d} si riportano dei grafici che riportano l'accuratezza ottenuta sui dati di \emph{test} sia in termini assoluti che in rapporto con l'accuratezza del modello senza vincolo sul \emph{budget} (\emph{score ratio}) relativamente ai \emph{dataset} più significativi.
I grafici qui omessi sono riportati nell'Appendice A.

Dai risultati ottenuti è possibile notare come per \emph{dataset} ``semplici'', o con una funzione di etichettatura sinusoidale con tante oscillazioni ma ben approssimabile da una superficie lineare, il metodo proposto risulta efficace, portando ad una riduzione significativa del numero di vettori di supporto senza pagare troppo in termini di accuratezza.
In altri casi, invece, con \emph{dataset} non necessariamente più difficili secondo le metriche F1 e F1v, il comportamento è diverso:
\begin{itemize}
    \item per piccole riduzioni di budget, come $80\%$ e $90\%$ del numero di vettori di supporto del modello tradizionale, la perdita in accuratezza è moderata;
    \item per valori di budget intermedi nel range considerato, tra il $40\%$ e il $70\%$ la perdita di accuratezza è più grave e marcata;
    \item per valori di budget piccoli, come $30\%$ o $40\%$, l'accuratezza è minore del modello senza budget ma comunque non proporzionale alla riduzione della dimensione del modello.
\end{itemize}
Sempre per i \emph{dataset} più difficili, si nota anche un'alta variabilità nei risultati ottenuti nel gruppo di \emph{dataset} generati con stessi parametri ma seme diverso.
\begin{figure}
    \begin{subfigure}{.5\textwidth}
        \centering
        \includegraphics[width=\textwidth]{img/2d/3.pdf}
    \end{subfigure}%
    \begin{subfigure}{.5\textwidth}
        \centering
        \includegraphics[width=\textwidth]{img/2d/4.pdf}
    \end{subfigure}%
    %
    \hfill
    %
    \begin{subfigure}{.5\textwidth}
        \centering
        \includegraphics[width=\textwidth]{img/2d/8.pdf}
    \end{subfigure}%
    \begin{subfigure}{.5\textwidth}
        \centering
        \includegraphics[width=\textwidth]{img/2d/9.pdf}
    \end{subfigure}%
    %
    \hfill
    %
    \begin{subfigure}{.5\textwidth}
        \centering
        \includegraphics[width=\textwidth]{img/2d/10.pdf}
    \end{subfigure}%
    \begin{subfigure}{.5\textwidth}
        \centering
        \includegraphics[width=\textwidth]{img/2d/12.pdf}
    \end{subfigure}%
     %
    \hfill
    %
    \begin{subfigure}{.5\textwidth}
        \centering
        \includegraphics[width=\textwidth]{img/2d/13.pdf}
    \end{subfigure}%
    \begin{subfigure}{.5\textwidth}
        \centering
        \includegraphics[width=\textwidth]{img/2d/14.pdf}
    \end{subfigure}%
     %
    \hfill
    %
    \begin{subfigure}{.5\textwidth}
        \centering
        \includegraphics[width=\textwidth]{img/2d/15.pdf}
    \end{subfigure}%
    \caption[Risultati su \emph{dataset} sintetici utilizzando la strategia 1.]{Risultati degli esperimenti su \emph{dataset} sintetici utilizzando la strategia 1. Ognuno dei grafici a sinistra indica l'andamento dell'accuratezza sui dati di \emph{test} al variare del \emph{budget}; ognuno dei grafici al centro indica lo \emph{score ratio} al variare del \emph{budget} (un simbolo ``x'' indica che il risolutore ha ritornato una soluzione non ottima); ognuno dei grafici a destra rappresenta uno dei tre \emph{dataset} utilizzati con le rispettive metriche di difficoltà.}
\label{fig:risultati_2d}
\end{figure}

% magari full_budget partiva già con "pochi" sv e ridurli porta ad accuracy molto più basse... ma non so

Analizzando il numero di vettori di supporto dei modelli tradizionali,~\Cref{fig:2d_dist_numsv}, è possibile notare come la maggior parte dei modelli abbia un numero ragionevole di vettori di supporto ma come una minoranza abbia invece un numero molto elevato.
\begin{figure}
    \centering
    \includegraphics[width=0.5\linewidth]{img/2d/numsv.pdf}
    \caption{Distribuzione del numero di vettori di supporto per i modelli classici addestrati sui dataset sintetici con la strategia 1.}
    \label{fig:2d_dist_numsv}
\end{figure}

\subsubsection{Risultati ottenuti utilizzando la strategia 2}
Utilizzando gli stessi \emph{dataset} dell'esperimento precedente, sono stati effettuati ulteriori esperimenti utilizzando la strategia 2.
In~\Cref{fig:2d_v2} si riporta l'andamento dell'accuratezza sui dati di \emph{test} al variare del \emph{budget} per ogni dataset.
\begin{figure}
    \begin{subfigure}{.5\textwidth}
        \centering
        \includegraphics[width=\textwidth]{img/2d_v2/4.pdf}
    \end{subfigure}%
    \begin{subfigure}{.5\textwidth}
        \centering
        \includegraphics[width=\textwidth]{img/2d_v2/5.pdf}
    \end{subfigure}
    %
    \hfill
    %
    \begin{subfigure}{.5\textwidth}
        \centering
        \includegraphics[width=\textwidth]{img/2d_v2/7.pdf}
    \end{subfigure}
    \begin{subfigure}{.5\textwidth}
        \centering
        \includegraphics[width=\textwidth]{img/2d_v2/8.pdf}
    \end{subfigure}%
    %
    \hfill
    %
    \begin{subfigure}{.5\textwidth}
        \centering
        \includegraphics[width=\textwidth]{img/2d_v2/12.pdf}
    \end{subfigure}
    \begin{subfigure}{.5\textwidth}
        \centering
        \includegraphics[width=\textwidth]{img/2d_v2/13.pdf}
    \end{subfigure}%
    %
    \hfill
    %
    \begin{subfigure}{.5\textwidth}
        \centering
        \includegraphics[width=\textwidth]{img/2d_v2/14.pdf}
    \end{subfigure}
    \begin{subfigure}{.5\textwidth}
        \centering
        \includegraphics[width=\textwidth]{img/2d_v2/15.pdf}
    \end{subfigure}%
\caption[Risultati su \emph{dataset} sintetici utilizzando la strategia 2.]{Esperimenti su \emph{dataset} sintetici 2D utilizzando la strategia 2. Ognuno dei grafici a sinistra indica l'andamento dell'accuratezza sui dati di \emph{test} al variare del \emph{budget}; ognuno dei grafici a destra rappresenta il \emph{dataset} utilizzato con le rispettive metriche di difficoltà.}
\label{fig:2d_v2}
\end{figure}
Dai risultati ottenuti si può notare vedere come anche per la maggior parte dei \emph{dataset} ``difficili'' si possa utilizzare un \emph{budget} stringente, tra il $7\%$ ed il $10\%$, con perdite accettabili di accuratezza.
Questo comportamento sembra in linea con l'osservazione fatta per l'esperimento precedente, dove per valori di budget stringenti ($30\%$ per la strategia 1) la perdita di accuratezza su questi \emph{dataset} non è così marcata.

Per \emph{dataset} la cui una superficie di separazione lineare approssima bene la funzione di etichettatura originale, la riduzione di \emph{budget} può essere molto importante, anche con solo l'$1\%$ del \emph{dataset} si ottiene accuratezza pari a quella ottenuta con \emph{budget} molto più alti.

% \section{Esperimenti su \emph{dataset} sintetici 3D}\label{sec:exp:synth_3d}
% Una piccola parte degli esperimenti è stata effettuata su \emph{dataset} in 3 dimensioni con 5600 dati di addestramento e 2400 di test.
% La strategia utilizzata è quella di impostare il \emph{budegt} in funzione del numero di vettori di supporto di un modello classico. 
% In~\Cref{fig:3d_exp} si possono vedere i risultati ottenuti e i \emph{dataset} utilizzati.
% \begin{figure}
%     \begin{subfigure}{\textwidth}
%         \centering
%         \includegraphics[width=\textwidth]{img/3d/1.pdf}
%     \end{subfigure}%
%     \hfill
%     \begin{subfigure}{\textwidth}
%         \centering
%         \includegraphics[width=\textwidth]{img/3d/2.pdf}
%     \end{subfigure}%
%     \hfill
%     \begin{subfigure}{\textwidth}
%         \centering
%         \includegraphics[width=\textwidth]{img/3d/3.pdf}
%     \end{subfigure}%
% \caption{Esperimenti su \emph{dataset} sintetici 3D con variazione del \emph{budget} rispetto ad un modello classico. A sinistra l'accuratezza sui dati di test al variare del budget; al centro lo ``score ratio'' al variare del budget; a destra il \emph{dataset} ridotto a due dimensioni utilizzando \emph{principal component analysis}.}
% \label{fig:3d_exp}
% \end{figure}
% Per limitare la quantità di tempo e risorse richiesti per eseguire questo tipo di esperimenti sono stati utilizzati solamente 3 dataset.

% Dai risultati ottenuti possiamo comunque notare come restrizioni di \emph{budget} significative risultino addirittura in migliori performance rispetto al modello nella formulazione classica. 
% Analizzando i modelli ottenuti però si nota come il numero di vettori di supporto nella formulazione classica sia molto elevato (5550, 5592, 5600), praticamente l'intero insieme di dati di addestramento.


\section{Esperimenti su \emph{dataset} di terze parti}\label{sec:exp:real_ds}
Utilizzando i \emph{dataset} descritti nella~\Cref{tab:uci_datasets} sono stati effettuati analoghi a quelli effettuati su dataset sintetici.

I risultati ottenuti utilizzando la strategia 1 sono visualizzabili in~\Cref{fig:TP_old_strategy}.
\begin{figure}
    \centering
    \includegraphics[width=1\linewidth]{img//TP/tp_old_strategy.pdf}
    \caption[Risultati su \emph{dataset} di terze parti utilizzando la strategia 1.]{Risultati esperimenti su \emph{dataset} di terze parti utilizzando la strategia 1. A sinistra l'accuratezza sui dati di test al variare del budget; a destra lo \emph{score ratio} al variare del budget.}
    \label{fig:TP_old_strategy}
\end{figure}
Possiamo notare come su questi \emph{dataset} tutte le riduzioni di \emph{budget} testate siano efficaci e non risultino in perdite di accuratezza significative.
Ogni modello \emph{budgeted SVC} è sostanzialmente equivalente come bontà al rispettivo modello classico.
L'accuratezza ottenuta sul \emph{dataset gisette} indica come nessuno dei modelli prodotti sia efficace a modellare quel problema; questo risultato non è preoccupante perché le caratteristiche di questi dati rendono il problema particolarmente difficile.
Questo \emph{dataset} è di dimensione molto ridotta (6000 elementi di addestramento) rispetto al numero degli attributi (5000), ed era stato introdotto in letteratura per valutare algoritmi di selezione degli attributi. 

Infine, utilizzando gli stessi \emph{dataset} si sono eseguiti degli esperimenti utilizzando la strategia 2; si riportano i risultati in~\Cref{fig:TP_new_strategy}
\begin{figure}
    \centering
    \includegraphics[width=0.5\linewidth]{img//TP/tp_new_strategy.pdf}
    \caption[Risultati su \emph{dataset} di terze parti utilizzando la strategia 2.]{Accuratezza al variare del \emph{budget} sui \emph{dataset} di terze parti utilizzando la strategia 2.}
    \label{fig:TP_new_strategy}
\end{figure}
Dai risultati di questi esperimenti, così come per gli analoghi esperimenti effettuati su \emph{dataset} sintetici, è possibile notare come per ogni \emph{dataset} sia possibile identificare una soglia di \emph{budget} per cui valori più alti non risultano in miglioramenti e per cui valori più bassi risultano in una diminuzione (più o meno repentina) di accuratezza.

\section{Comparazione con altri metodi}\label{sec:comparazione_metodi}
Per meglio inquadrare i risultati ottenuti dal metodo proposto in questa tesi, son stati ripetuti gli esperimenti effettuati utilizzando delle implementazioni di metodi proposti in letteratura e visti nel~\Cref{chap:sparse_svc}, in particolare:
\begin{itemize}
    \item \emph{BSGD SVM} esposto in~\cite{2012_bsgd}. Risolutore pensato per addestramento \emph{on-line} basato su discesa del gradiente che mantiene una dimensione massima dell'insieme dei vettori di supporto utilizzando la strategia di unione o rimozione.
    \item \emph{NSSVM} esposto in~\cite{2020_sparse_svm}. Risolutore basato su \emph{Newton method} che minimizza il numero di vettori di supporto tramite una funzione di costo adatta.
    % \item \emph{LIB IRWLS} esposto in~\cite{LIBIRWLS}. Risolutore parallelo basato su un approssimazione \emph{greedy} della matrice \emph{kernel} e utilizzo dell'algoritmo IRWLS~\cite{IRWLS}
\end{itemize}
Utilizzando gli stessi \emph{dataset} sintetici generati con i parametri nelle~\Cref{tab:parametri_ds_sin,tab:parametri_ds_pacman} e utilizzando \emph{5-fold cross-validation grid search} con una griglia di parametri adatti (\Cref{tab:gridsearch_comparazioni}), viene misurata l'accuratezza sui dati di \emph{test} per gli stessi valori di \emph{budget} utilizzati negli esperimenti precedenti.
\begin{table}
    \centering
    \begin{tabular}{ccccc}
        \toprule
        Algoritmo & $C$ & \emph{Kernel} & $\gamma$ & d \\
        \midrule
        \multirow{2}{*}{BSGD}   & \multirow{2}{*}{/}  & Gaussiano   & [0.001, 0.01, 0.1, 1, 10]   & /\\
                                      \cline{3-5}
                                &   & Polinomiale & / & [2, 5, 10] \\
        \hline
        NSSVM   & / & / & / & / \\
        % \hline
        % IRWLS   & [0.01, 0.1, 1, 10]  & Gaussiano & [0.001, 0.01, 0.1, 1, 10] & / \\
        \bottomrule
    \end{tabular}
    \caption{Parametri \emph{grid search} per gli algoritmi presenti in letteratura.}
    \label{tab:gridsearch_comparazioni}
\end{table}
Nelle~\Cref{fig:comp_old} si possono vedere i valori di accuratezza ottenuti utilizzando la strategia 1 confrontati ai risultati ottenuti con \emph{budgeted SVC}.
\begin{figure}
    \begin{subfigure}{.5\textwidth}
        \centering
        \includegraphics[width=\textwidth]{img/comp_old/3.pdf}
    \end{subfigure}%
    \begin{subfigure}{.5\textwidth}
        \centering
        \includegraphics[width=\textwidth]{img/comp_old/4.pdf}
    \end{subfigure}
    %
    \hfill
    %
    \begin{subfigure}{.5\textwidth}
        \centering
        \includegraphics[width=\textwidth]{img/comp_old/8.pdf}
    \end{subfigure}
    \begin{subfigure}{.5\textwidth}
        \centering
        \includegraphics[width=\textwidth]{img/comp_old/9.pdf}
    \end{subfigure}%
    %
    \hfill
    %
    \begin{subfigure}{.5\textwidth}
        \centering
        \includegraphics[width=\textwidth]{img/comp_old/10.pdf}
    \end{subfigure}
    \begin{subfigure}{.5\textwidth}
        \centering
        \includegraphics[width=\textwidth]{img/comp_old/12.pdf}
    \end{subfigure}%
    %
    \hfill
    %
    \begin{subfigure}{.5\textwidth}
        \centering
        \includegraphics[width=\textwidth]{img/comp_old/13.pdf}
    \end{subfigure}
    \begin{subfigure}{.5\textwidth}
        \centering
        \includegraphics[width=\textwidth]{img/comp_old/14.pdf}
    \end{subfigure}
        %
    \hfill
    %
    \begin{subfigure}{.5\textwidth}
        \centering
        \includegraphics[width=\textwidth]{img/comp_old/15.pdf}
    \end{subfigure}
\caption[Risultati su \emph{dataset} sintetici utilizzando strategia 1 in confronto ad altri metodi.]{Risultati più significativi ottenuti su \emph{dataset} sintetici 2D utilizzando la strategia 1, analogo ai risultati in~\Cref{fig:risultati_2d} ma con una curva per ogni metodo utilizzato: la proposta \emph{Budgeted SVC}, i metodi \emph{NSSVM} e \emph{BSGD}.}
\label{fig:comp_old}
\end{figure}   
Nelle~\Cref{fig:comp_new} si possono vedere i valori di accuratezza ottenuti utilizzando la strategia 1 confrontati ai risultati ottenuti con \emph{budgeted SVC}.
\begin{figure}
    \begin{subfigure}{.5\textwidth}
        \centering
        \includegraphics[width=\textwidth]{img/comp_new/4.pdf}
    \end{subfigure}%
    \begin{subfigure}{.5\textwidth}
        \centering
        \includegraphics[width=\textwidth]{img/comp_new/5.pdf}
    \end{subfigure}
    %
    \hfill
    %
    \begin{subfigure}{.5\textwidth}
        \centering
        \includegraphics[width=\textwidth]{img/comp_new/7.pdf}
    \end{subfigure}
    \begin{subfigure}{.5\textwidth}
        \centering
        \includegraphics[width=\textwidth]{img/comp_new/8.pdf}
    \end{subfigure}%
    %
    \hfill
    %
    \begin{subfigure}{.5\textwidth}
        \centering
        \includegraphics[width=\textwidth]{img/comp_new/12.pdf}
    \end{subfigure}
    \begin{subfigure}{.5\textwidth}
        \centering
        \includegraphics[width=\textwidth]{img/comp_new/13.pdf}
    \end{subfigure}%
    %
    \hfill
    %
    \begin{subfigure}{.5\textwidth}
        \centering
        \includegraphics[width=\textwidth]{img/comp_new/14.pdf}
    \end{subfigure}
    \begin{subfigure}{.5\textwidth}
        \centering
        \includegraphics[width=\textwidth]{img/comp_new/15.pdf}
    \end{subfigure}
\caption[Risultati su \emph{dataset} sintetici utilizzando strategia 2 in confronto ad altri metodi.]{Risultati più significativi ottenuti su \emph{dataset} sintetici 2D utilizzando la strategia 2, analogo ai risultati in~\Cref{fig:2d_v2} ma con una curva per ogni metodo utilizzato: la proposta \emph{Budgeted SVC}, i metodi \emph{NSSVM} e \emph{BSGD}.}
\label{fig:comp_new}
\end{figure}   

\backmatter
    \chapter{Conclusioni e sviluppi futuri}
\label{chap:conclusioni}
Lo studio di algoritmi di apprendimento automatico in grado di produrre modelli efficienti in spazio e calcolo è interessante per diversi motivi. 
In ambienti caratterizzati da una scarsità di risorse, utilizzare modelli di dimensioni ridotte potrebbe essere una necessità.
In genere, un modello che richiede meno spazio e meno risorse di calcolo rispetto a un modello tradizionale consuma anche meno energia. Questa caratteristica potrebbe essere fondamentale nel caso di dispositivi alimentati a batteria e potrebbe comunque essere apprezzata nel caso di macchine tradizionali. 

I modelli SVM sono stati utilizzati negli anni per risolvere diversi problemi, esibendo in molti casi prestazioni competitive. Nella formulazione classica non sono però adatti a contesti con limitate risorse di memorizzazione e calcolo.

In questa tesi è stata analizzata la proposta \emph{budgeted SVC}, che consente di addestrare classificatori binari con un ristretto numero di vettori di supporto.
Gli esperimenti effettuati su diversi \emph{dataset} sintetici evidenziano come \emph{budgeted SVC} sia promettente. 
Le motivazioni sono esposte nell'elenco seguente.
\begin{itemize}
    \item Per alcuni \emph{dataset} identificati come facili, per cui le metriche di complessità F1 e F1v hanno valori bassi, la riduzione in termini di spazio può essere significativa e non penalizzante in termini di accuratezza.
    \item Per altri \emph{dataset} identificati come difficili, per cui le metriche di complessità F1 e F1v hanno valori alti, l'accuratezza subisce un peggioramento significativo al diminuire del \emph{budget}, fino al punto in cui per riduzioni importanti l'accuratezza ritorna a essere buona. Tuttavia, per i modelli peggiori le soluzioni restituite dal risolutore non sono ottime.
    \item Esprimendo il \emph{budget} come percentuale della dimensione del \emph{dataset}, si nota come valori tra il $5\%$ ed il $7\%$ della dimensione del \emph{dataset} possano essere delle buone scelte per risparmiare spazio senza sacrificare troppo in termini di accuratezza.
\end{itemize}

Gli esperimenti effettuati su \emph{dataset} di terze parti sono anche più promettenti dei precedenti risultati:
\begin{itemize}
    \item utilizzando la strategia 1, tutti i modelli con \emph{budget} mostrano un'accuratezza pari o leggermente inferiore rispetto all'accuratezza del corrispettivo modello classico;
    \item utilizzando la strategia 2, superato un certo valore di \emph{budget} soglia, la perdita in accuratezza risulta proporzionale alla riduzione di \emph{budget}.
\end{itemize}

Il confronto con altri metodi, utilizzando la strategia 1, evidenzia come \emph{budgeted SVC} sia in alcuni casi competitivo su \emph{dataset} semplici ma come sia invece maggiormente sensibile alle riduzioni di \emph{budget} su \emph{dataset} complessi.
Dai risultati ottenuti utilizzando la strategia 2, si nota che gli altri metodi considerati, NSSVM e BSGD SVM,  producono in genere modelli più accurati per valori di \emph{budget} minimi, tra l'$1\%$ e il $5\%$ della dimensione del \emph{dataset} di addestramento.

Per risparmiare ulteriore spazio nella memorizzazione di un modello SVC, anche nella formulazione classica, è stata proposta una procedura di memorizzazione compressa.
Relativamente ai modelli prodotti, si evidenza come questa rappresentazione consenta in parecchi casi una riduzione significativa dello spazio richiesto per memorizzare il modello.

Nonostante i risultati ottenuti siano un buon punto di partenza, la formulazione \emph{budgeted SVC} potrebbe essere studiata ulteriormente, cercando di espandere, migliorare o chiarire alcuni aspetti, elencati di seguito.
\begin{itemize}
    % \item Si potrebbe verificare se la variabilità ottenuta su diversi \emph{dataset} generati con gli stessi parametri sia effettivamente dovuta al fatto che il risolutore non fornisca una soluzione ottima. 
    \item Si potrebbero utilizzare \emph{dataset} sintetici di dimensione maggiore e/o \emph{dataset} con un più alto numero di attributi. Si potrebbero utilizzare ulteriori \emph{dataset} di terze parti con caratteristiche diverse rispetto a quelli già considerati.
    \item Si potrebbero incorporare delle proposte presentate in letteratura, per esempio per limitare l'effetto del rumore o per rendere il problema trattabile con altri metodi risolutivi.
    \item Si potrebbe investigare un'eventuale relazione tra \emph{budget} e rumore, per capire se una quantità molto stringente di \emph{budget} corrisponda ad una maggior robustezza rispetto a dati erroneamente etichettati.
\end{itemize}



\bibliographystyle{unsrt}
\bibliography{bibliografia}
\addcontentsline{toc}{chapter}{Bibliografia}

\end{document}


 
